\documentclass{article}

\setlength{\oddsidemargin}{0in}
\setlength{\textwidth}{6in}
\setlength{\topmargin}{-0.1in}
\setlength{\textheight}{8.2in}

%%%%%%%%%%%%%  IMPORT MACRO FILES AS NEEDED %%%%%%%%%%%
\usepackage{amsgen,amsmath,amstext,amsbsy,amsopn,amssymb,amsthm,stackengine}
\usepackage{array, nicefrac, mathtools}
\usepackage{verbatim}
\usepackage{hyperref}
\usepackage{float,relsize,setspace,enumitem,pbox,cleveref,multicol,multirow}
\usepackage{multido}
\usepackage{bbding} % Has a checkmark symbol reachable through \Checkmark
\usepackage{tikz,mdframed}
% \usepackage{circuitikz}

% Theorems, definitions, equations, lemmas
\newtheorem{thm}{Theorem}[section]
\newtheorem{prop}[thm]{Proposition}
\newtheorem{lem}[thm]{Lemma}
\newtheorem{cor}[thm]{Corollary}
\newtheorem{defn}{Definition}
\newtheorem{rem}[thm]{Remark}
\numberwithin{equation}{section}
\newtheorem*{defn*}{Definition} % Theorem environments with no numbering
\newtheorem*{prop*}{Proposition}
\newtheorem*{thm*}{Theorem}
\theoremstyle{definition}
\newtheorem*{fact}{Fact}

% For negation and quantifiers in Discrete Math
\newcommand{\shortsim}{\raise.17ex\hbox{$\scriptstyle \sim$}}
\renewcommand{\neg}{\shortsim}
\renewcommand{\nexists}{\neg(\exists}
\newcommand{\nequiv}{\ensuremath{\not\equiv}}

\newcommand{\myline}[1]{\underline{\hspace{#1}}}
\newcounter{parts}
\newcounter{problems}[parts]
\newcounter{questions}[problems]
\newcounter{subquestions}[questions]
\newcommand{\hwpart}[1]{
  \stepcounter{parts}
  \noindent\makebox[\textwidth]{\LARGE \bf Part \arabic{parts} - #1}
  \\
}
\newcommand{\problem}[2]{\stepcounter{problems}
  {\Large \bf \noindent Problem \arabic{problems}: #1 \marginpar{[Total #2 pts]} \\[0.3cm]}}
\newcommand{\question}[2]{\stepcounter{questions}
  {\large (\alph{questions}) #1 \marginpar{[#2 pts]} \\[.3cm]}}
\newcommand{\subquestion}[2]{\stepcounter{subquestions}
  {\hspace{10pt}\emph{(\roman{subquestions}) #1 \marginpar{[#2 pts]} }\\[.3cm]}}

% Solution formatting
\newcommand{\solution}[1]{{\color{red}{#1}}}
% Some standard centering and italicization of text.
\newcommand{\frontrowcenter}[1]{\begin{center}{\em \Large  #1  }\end{center}}

% A blank page
\newcommand{\blankpage}{
\clearpage
\vspace*{\fill}
\begin{minipage}{\textwidth}
  \Large \textbf{THIS PAGE INTENTIONALLY LEFT BLANK}\\
\end{minipage}
\vfill % equivalent to \vspace{\fill}
\clearpage
}

\newcommand{\answerspace}[1]{
  \begin{center}
    \textbf{BEGIN YOUR ANSWER BELOW THIS LINE} \\ \hrulefill \vspace{#1} \\ \hrulefill
  \end{center}
}

\newcommand{\answerspacefullpage}{
  \begin{center}
    \textbf{BEGIN YOUR ANSWER BELOW THIS LINE} \\ \hrulefill \pagebreak
  \end{center}
}

\newcommand{\additionalanswerspace}[1]{
  \begin{center}
    \textbf{CONTINUE YOUR ANSWER BELOW THIS LINE } \\ \hrulefill \vspace{#1} \\ \hrulefill
  \end{center}
}

\newcommand{\additionalanswerspacefullpage}{
  \begin{center}
    \textbf{CONTINUE YOUR ANSWER BELOW THIS LINE} \\ \hrulefill \pagebreak
  \end{center}
}

\newcommand{\freespace}[1]{
  \begin{center}
    \large \textbf{SCRAP SPACE BELOW} \\
    \hrulefill
    \pagebreak
  \end{center}
}

% Centered line
\newcommand{\mycenterline}[1]{
  \begin{center}
    \myline{#1}
  \end{center}
}

% Space for T/F:
\newcommand{\tfline}{\myline{.5cm}}

% For quick parenthesized and italicized point annotation.
\newcommand{\pts}[1]{{\em (#1 pts)}}
\newcommand{\onept}{{\em (1 pt)}}

% \item environments coupled with a line at the end, for students to write T and F in.
\newcommand{\tfitem}[1]{\item #1 \null\hfill \framebox(25,25){} \\ \hdashrule{0.95\textwidth}{1pt}{2pt}}
\newcommand{\setitem}[1]{\tfitem{$\curlybraces{#1}$} }
\newcommand{\lineitem}[2]{\item #1 \null \hfill \myline{#2}}

% Some circles and squares for students to fill in.
\newcommand{\whitecircle}[1]{\tikz[baseline=-0.5ex]\draw[black, radius=#1] (0,0) circle ;}
\newcommand{\whitesquare}[1]{\tikz\draw[black] (0,0) rectangl#1, #1) ;}

% Emphasis
\newcommand{\F}{$\mathbf{F}$}
\newcommand{\T}{$\mathbf{T}$}
\newcommand{\False}{\textbf{False}}
\newcommand{\false}{\textbf{false}}
\newcommand{\True}{\textbf{True}}
\newcommand{\true}{\textbf{true}}
\newcommand{\makered}[1]{\textcolor{red}{#1}}
\newcommand{\Rbbst}{\textcolor{red}{Red}-black tree}
\newcommand{\rbbst}{\textcolor{red}{red}-black tree}

\newcommand{\homeworkdata}[4]{
  \begin{mdframed}[linewidth=1pt]
    \noindent\makebox[\textwidth]{\LARGE \bf #1, #2 }
    \\\\
    \noindent\makebox[\textwidth]{\Large \bf  Homework \##3 }
    \\\\
    \noindent\makebox[\textwidth]{\large \bf  Due: #4}
    \\\\
    \noindent\makebox[\textwidth]{\large \bf Homework will not be accepted late}
  \end{mdframed}
  \vspace{40pt}
}

\usepackage{circuitikz}

\setlength{\parindent}{0em}
\setlength{\itemindent}{.5in}

\newcommand{\poneanswer}{%
}
\newcommand{\ptwoanswer}{%
}
\newcommand{\pthreeanswer}{%
}
\newcommand{\pfouranswer}{%
}
\newcommand{\pfiveanswer}{%
}
\newcommand{\psixanswer}{%
}
% \include{solutions}

%%%%%%%%%%%%%%%%%%%%%%%%%%%%%%%%%%%%%%%%%%%%%
%
%  STUDENTS - Your homework begins here.
%
%%%%%%%%%%%%%%%%%%%%%%%%%%%%%%%%%%%%%%%%%%%%%

\begin{document}
\pagestyle{empty}

\homeworkdata{CMSC 250}{Fall 2022}{1}{Sunday 11 Sept.\ 11:59pm}

{\Large \bf
  \begin{center}
  IMPORTANT
  \end{center}

  You can write your answers on any paper, either this paper
  or blank paper, or write your answer in Latex (template of this homework can be downloaded through ELMS).
  
  When you upload your document to Gradescope, make sure you tag your questions.

  \begin{center}
    YOU WILL NEED TO TAG YOUR PROBLEMS!!!
  \end{center}

  For homework 1, problems that are not tagged will receive a 20\% penalty. For later homework, problems which are not correctly found will not be graded, this is a zero-tolerance policy. 

  \begin{center}
    IF YOU ARE WORRIED...
  \end{center}

  If you have concerns about tagging your problems,
  homework 0 is designed for testing how to tag. For example if you've never done it before, we strongly suggest you drop by office hours and do it with a TA present so they can help you through the process,
  just to see how it works. In addition, Gradescope has a tutorial: \url{https://help.gradescope.com/article/ccbpppziu9-student-submit-work#submitting_a_pdf}


}

\pagebreak
\pagebreak

\problem{Statement}{20}%
For each of the following English sentences, determine whether it is a logical statement. If it is not a statement, you must give a brief explanation.

\bigskip
\question{\textit{there is free food every Monday in 2010.}}{5}%
\bigskip
\textbf{Answer: True}\\\\
\bigskip
\bigskip
\bigskip
\bigskip

\question{\textit{The moon tonight is beautiful.}}{5}%

\bigskip
\textbf{Answer: False}\\\\
\textbf{Explanation if this is not a statement:\\\\
It's subjective, you can't say that it's true or false}
\bigskip
\bigskip
\bigskip
\bigskip

\question{\textit{Now, this wall is pink.}}{5}%

\bigskip
\textbf{Answer: True}\\\\
\bigskip
\bigskip
\bigskip
\bigskip

\question{\textit{ Life in the world is but a big dream;\\
 I will not spoil it by any labour or care.}}{5}%

\bigskip
\textbf{Answer: False}\\\\
\textbf{Explanation if this is not a statement:\\\\
The second part seems to be a statement but the first part is not a statement, it's an opinion.}
\bigskip
\bigskip
\bigskip
\bigskip

\poneanswer

\pagebreak



\problem{Converting Logic and English}{30}%

English can be more ambiguous than mathematics, so we’ll take the ambiguity into account when grading these questions.
\bigskip

\question{Logic to English}{15}
Use the following logical symbols:\\

\begin{itemize}
    \item $p$ = Apples are red.
    
    \item $q$ = The sky is blue.

    \item $r$ = Cliff is happy.
\end{itemize}
Translate each propositional logic expression into an English sentence, do not simplify:\\
\begin{enumerate}[label = (\alph*)]
	\item $p \land q$: Apples are red and the sky is blue.
	\item $\neg(p \lor r)$: It is not true that apples are red or that Cliff is happy.
	\item $r \land ( p \lor q)$: Cliff is happy and it is true that apples are red or that the sky is blue.\\

\end{enumerate}

\question{English to logic}{15}
Use the following logical symbols:\\

\begin{itemize}
    \item $p$ = Apples are red.
    
    \item $q$ = The sky is blue.

    \item $r$ = Cliff is happy.
\end{itemize}
Translate each English sentence into a propositional logic expression. Note that the
truth value of the statement is irrelevant - we want the direct translation, no matter whether
the statement is true or false:\\
\begin{enumerate}[label = (\alph*)]
	\item If apples are red, then the sky is blue.: $p \to q$
	\item Cliff is happy if apples are red and the sky is blue.: $(p \land q) \to r$
	\item Cliff is not happy if and only if apples are not red: $\lnot r \Leftrightarrow \lnot p$\\
\end{enumerate}

\ptwoanswer

\pagebreak

\problem{Truth Table}{10}%
\question{\textit{Fill in the truth table for $r \land (\neg(\neg p \land q))$}}{4}.
\begin{center}
	\begin{table}[H] 
		\large 
		\setlength{\tabcolsep}{15pt}
%		\renewcommand{\arraystretch}{1.2}
		\begin{tabular}{c|c|c|c|c|c|c|c|} \hline 
		0.&	$p$ & $q$ & $r$ & $\neg p$ & $(\neg p \land q)$  & $\neg(\neg p \land q)$ & $r \land (\neg(\neg p \land q))$\\ \hline
		1.&	0 & 0 & 0 & 1 & 0 & 1 & 0 \\ \hline 		
		2.&	0 & 0 & 1 & 1 & 0 & 1 & 1 \\ \hline 		
		3.&	0 & 1 & 0 & 1 & 1 & 0 & 0 \\ \hline 		
		4.&	0 & 1 & 1 & 1 & 1 & 0 & 0 \\ \hline
		5.&	1 & 0 & 0 & 0 & 0 & 1 & 0 \\ \hline 		
		6.&	1 & 0 & 1 & 0 & 0 & 1 & 1 \\ \hline
		7.&	1 & 1 & 0 & 0 & 0 & 1 & 0 \\ \hline 		
		8.&	1 & 1 & 1 & 0 & 0 & 1 & 1 \\ \hline												
		\end{tabular}
	\end{table}
\end{center}


\question{\textit{Which rows make the statement false?}}{3}%
\textbf{Answer: 1, 3, 4, 5, 7}

\bigskip

\question{\textit{Which rows make the statement true?}}{3}%
\textbf{Answer: 2, 6, 8}

\pthreeanswer

\pagebreak

\problem{Converse, Inverse, Contrapositive}{25}

\question{Write the converse, inverse and contrapositive for the following statement. Then state whether each statement must be true or not and why.}{15}
\begin{center}
If I jump into the lake, I will get wet. 
\end{center}
\textbf{Converse: If I get wet, I will have jumped into the lake.}\\

\bigskip
\bigskip

\textbf{Inverse: If I don't jump into the lake, I won't get wet.}\\

\bigskip
\bigskip

\textbf{Contrapositive: If I don't get wet, I will not have jumped into the lake.}\\

\bigskip
\bigskip
    
\question{Now come up with a statement in the form $p \Rightarrow q$ whose converse, inverse and contrapositive are all true statements, and $p, q$ are two different statements.}{10}
\textbf{Answer: If an integer is even, it is not odd.}
%
\pfouranswer

\pagebreak

\problem{Necessary and Sufficient}{15}%
What are the necessary and sufficient conditions of "apples are red" in these statements? If there is no necessary or sufficient condition, specify which one(s) the statement is missing.

\question{\textit{If apples are red, then the sky is blue.}}{5}%
\textbf{Answer: The sky being blue is a necessary condition for apples being red. There is no sufficient condition, though.}

\bigskip
\bigskip
\bigskip
\bigskip
\bigskip

\question{\textit{Cliff is not happy if and only if apples are red}}{5}%
\textbf{Answer: Cliff not being happy is both necessary and sufficient for apples being red.}

\bigskip
\bigskip
\bigskip
\bigskip
\bigskip

\question{\textit{Apples are red only if the sky is blue.}}{5}%
\textbf{Answer: The sky being blue is necessary for apples being red. There is no sufficient condition, though.}

\bigskip
\pfiveanswer

\pagebreak



\end{document}

%%% Local Variables:
%%% mode: latex
%%% TeX-master: t
%%% End:

