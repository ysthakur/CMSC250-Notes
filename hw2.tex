\documentclass[leqno]{article}

\setlength{\oddsidemargin}{0in}
\setlength{\textwidth}{6in}
\setlength{\topmargin}{-0.1in}
\setlength{\textheight}{8.2in}

%%%%%%%%%%%%%  IMPORT MACRO FILES AS NEEDED %%%%%%%%%%%
\usepackage{amsgen,amsmath,amstext,amsbsy,amsopn,amssymb,amsthm,stackengine}
\usepackage{array, nicefrac, mathtools}
\usepackage{verbatim}
\usepackage{hyperref}
\usepackage{float,relsize,setspace,enumitem,pbox,cleveref,multicol,multirow}
\usepackage{multido}
\usepackage{bbding} % Has a checkmark symbol reachable through \Checkmark
\usepackage{tikz,mdframed}
% \usepackage{circuitikz}

% Theorems, definitions, equations, lemmas
\newtheorem{thm}{Theorem}[section]
\newtheorem{prop}[thm]{Proposition}
\newtheorem{lem}[thm]{Lemma}
\newtheorem{cor}[thm]{Corollary}
\newtheorem{defn}{Definition}
\newtheorem{rem}[thm]{Remark}
\numberwithin{equation}{section}
\newtheorem*{defn*}{Definition} % Theorem environments with no numbering
\newtheorem*{prop*}{Proposition}
\newtheorem*{thm*}{Theorem}
\theoremstyle{definition}
\newtheorem*{fact}{Fact}

% For negation and quantifiers in Discrete Math
\newcommand{\shortsim}{\raise.17ex\hbox{$\scriptstyle \sim$}}
\renewcommand{\neg}{\shortsim}
\renewcommand{\nexists}{\neg(\exists}
\newcommand{\nequiv}{\ensuremath{\not\equiv}}

\newcommand{\myline}[1]{\underline{\hspace{#1}}}
\newcounter{parts}
\newcounter{problems}[parts]
\newcounter{questions}[problems]
\newcounter{subquestions}[questions]
\newcommand{\hwpart}[1]{
	\stepcounter{parts}
	\noindent\makebox[\textwidth]{\LARGE \bf Part \arabic{parts} - #1}
	\\
}
\newcommand{\problem}[2]{\stepcounter{problems}
	{\Large \bf \noindent Problem \arabic{problems}: #1 \marginpar{[Total #2 pts]} \\[0.3cm]}}
\newcommand{\question}[2]{\stepcounter{questions}
	{\large (\alph{questions}) #1 \marginpar{[#2 pts]} \\[.3cm]}}
\newcommand{\subquestion}[2]{\stepcounter{subquestions}
	{\hspace{10pt}\emph{(\roman{subquestions}) #1 \marginpar{[#2 pts]} }\\[.3cm]}}

% Solution formatting
\newcommand{\solution}[1]{{\color{red}{#1}}}
% Some standard centering and italicization of text.
\newcommand{\frontrowcenter}[1]{\begin{center}{\em \Large  #1  }\end{center}}

% A blank page
\newcommand{\blankpage}{
	\clearpage
	\vspace*{\fill}
	\begin{minipage}{\textwidth}
		\Large \textbf{THIS PAGE INTENTIONALLY LEFT BLANK}\\
	\end{minipage}
	\vfill % equivalent to \vspace{\fill}
	\clearpage
}

\newcommand{\answerspace}[1]{
	\begin{center}
		\textbf{BEGIN YOUR ANSWER BELOW THIS LINE} \\ \hrulefill \vspace{#1} \\ \hrulefill
	\end{center}
}

\newcommand{\answerspacefullpage}{
	\begin{center}
		\textbf{BEGIN YOUR ANSWER BELOW THIS LINE} \\ \hrulefill \pagebreak
	\end{center}
}

\newcommand{\additionalanswerspace}[1]{
	\begin{center}
		\textbf{CONTINUE YOUR ANSWER BELOW THIS LINE } \\ \hrulefill \vspace{#1} \\ \hrulefill
	\end{center}
}

\newcommand{\additionalanswerspacefullpage}{
	\begin{center}
		\textbf{CONTINUE YOUR ANSWER BELOW THIS LINE} \\ \hrulefill \pagebreak
	\end{center}
}

\newcommand{\freespace}[1]{
	\begin{center}
		\large \textbf{SCRAP SPACE BELOW} \\
		\hrulefill
		\pagebreak
	\end{center}
}

% Centered line
\newcommand{\mycenterline}[1]{
	\begin{center}
		\myline{#1}
	\end{center}
}

% Space for T/F:
\newcommand{\tfline}{\myline{.5cm}}

% For quick parenthesized and italicized point annotation.
\newcommand{\pts}[1]{{\em (#1 pts)}}
\newcommand{\onept}{{\em (1 pt)}}

% \item environments coupled with a line at the end, for students to write T and F in.
\newcommand{\tfitem}[1]{\item #1 \null\hfill \framebox(25,25){} \\ \hdashrule{0.95\textwidth}{1pt}{2pt}}
\newcommand{\setitem}[1]{\tfitem{$\curlybraces{#1}$} }
\newcommand{\lineitem}[2]{\item #1 \null \hfill \myline{#2}}

% Some circles and squares for students to fill in.
\newcommand{\whitecircle}[1]{\tikz[baseline=-0.5ex]\draw[black, radius=#1] (0,0) circle ;}
\newcommand{\whitesquare}[1]{\tikz\draw[black] (0,0) rectangl#1, #1) ;}

% Emphasis
\newcommand{\F}{$\mathbf{F}$}
\newcommand{\T}{$\mathbf{T}$}
\newcommand{\False}{\textbf{False}}
\newcommand{\false}{\textbf{false}}
\newcommand{\True}{\textbf{True}}
\newcommand{\true}{\textbf{true}}
\newcommand{\makered}[1]{\textcolor{red}{#1}}
\newcommand{\Rbbst}{\textcolor{red}{Red}-black tree}
\newcommand{\rbbst}{\textcolor{red}{red}-black tree}

\newcommand{\homeworkdata}[4]{
	\begin{mdframed}[linewidth=1pt]
		\noindent\makebox[\textwidth]{\LARGE \bf #1, #2 }
		\\\\
		\noindent\makebox[\textwidth]{\Large \bf  Homework \##3 }
		\\\\
		\noindent\makebox[\textwidth]{\large \bf  Due: #4}
		\\\\
		\noindent\makebox[\textwidth]{\large \bf Homework will not be accepted late}
	\end{mdframed}
	\vspace{40pt}
}

\usepackage{circuitikz}

\setlength{\parindent}{0em}
\setlength{\itemindent}{.5in}

\newcommand{\poneanswer}{%
}
\newcommand{\ptwoanswer}{%
}
\newcommand{\pthreeanswer}{%
}
\newcommand{\pfouranswer}{%
}
\newcommand{\pfiveanswer}{%
}
\newcommand{\psixanswer}{%
}
% \include{solutions}

%%%%%%%%%%%%%%%%%%%%%%%%%%%%%%%%%%%%%%%%%%%%%
%
%  STUDENTS - Your homework begins here.
%
%%%%%%%%%%%%%%%%%%%%%%%%%%%%%%%%%%%%%%%%%%%%%

\begin{document}
	\pagestyle{empty}
	
	\homeworkdata{CMSC 250}{Fall 2022}{2}{Sunday 18 Sept.\ 11:59pm}
	
	{\Large \bf
		\begin{center}
			IMPORTANT
		\end{center}
		
		You can write your answers on any paper, either this paper
		or blank paper, or write your answer in Latex (template of this homework can be downloaded through ELMS).
		
		When you upload your document to Gradescope, make sure you tag your questions.
		
		\begin{center}
			YOU WILL NEED TO TAG YOUR PROBLEMS!!!
		\end{center}
		
		Problems which are not correctly found will not be graded, this is a zero-tolerance policy. 
		
		\begin{center}
			IF YOU ARE WORRIED...
		\end{center}
		
		If you have concerns about tagging your problems,
		We strongly suggest you drop by office hours and do it with a TA present so they can help you through the process,
		just to see how it works. In addition, Gradescope has a tutorial: \url{https://help.gradescope.com/article/ccbpppziu9-student-submit-work#submitting_a_pdf}
		
		
	}
	
	\pagebreak
	
	\problem{Equivalence: Rules}{16}%
	Show the following statements are equivalent using laws of equivalence. You should only use rules taught in this course. Show your work step by step, and label the rules that you use for each step. For Latex, you can try to use align* for aligned equations.  
	
	\bigskip
	
	\question{$(p \Rightarrow \neg q) \land (p \Rightarrow \neg r) \equiv \neg (p \land (q \lor r))$}{8}
	
	\textbf{Answer:}
	\begin{align*}
	    (p \Rightarrow \neg q) \land (p \Rightarrow \neg r) &\equiv (\lnot p \lor \neg q) \land (\lnot p \lor \neg r) & (\text{Definition of implication}) \\
	    &\equiv \lnot (p \land q) \land \lnot (p \land r) & (\text{De Morgan's}) \\
	    &\equiv \lnot ((p \land q) \lor (p \land r)) & (\text{De Morgan's}) \\
	    &\equiv \lnot (p \land (q \lor r) & (\text{Distributive})
	\end{align*}
	\bigskip
	\bigskip
	\bigskip
	\bigskip
	\bigskip
	\bigskip
	\bigskip
	\bigskip
	\bigskip
	\bigskip
	
	\question{$(p \land q) \Rightarrow (p \lor q) \equiv 1$}{8}
	
	\textbf{Answer:}
	\begin{align*}
	    (p \land q) \Rightarrow (p \lor q) &\equiv \lnot (p \land q) \lor (p \lor q) & (\text{Definition of implication}) \\
	    &\equiv 1 & (\text{Negation})
	\end{align*}
	\bigskip
	\bigskip
	\bigskip
	\bigskip
	\bigskip
	\bigskip
	\bigskip
	\bigskip
	\bigskip
	\bigskip
	
	
	
	\pagebreak
	
	\problem{Equivalence: Tables}{10}%
	Show whether the following statements are equivalent or not using truth tables. Feel free to add columns to show your work, but the given columns must be present in your final answer. We can award partial credit if work is shown.
	
	\bigskip
	
	\question{$(p \Rightarrow (q \lor r)) \equiv ((p \land \neg q) \Rightarrow q)$}{5}
	
	\textbf{Answer:}
	
	\begin{table*}[h!]
		\LARGE
		\begin{tabular}{c|c|c|c|c|c|c}
			$p$ & $q$ & $r$ & $q \land r$ & $p \land \lnot q$ & $(p \Rightarrow (q \lor r))$ & $(p \land \neg q) \Rightarrow q$ \\ \hline
			0 & 0 & 0 & 0 & 0 & 0 & 0 \\\hline
			0 & 0 & 1 & 0 & 0 & 0 & 0 \\\hline
			0 & 1 & 0 & 0 & 0 & 0 & 0 \\\hline
			0 & 1 & 1 & 1 & 0 & 0 & 0 \\\hline
			1 & 0 & 0 & 0 & 1 & 0 & 0 \\\hline
			1 & 0 & 1 & 0 & 1 & 0 & 0 \\\hline
			1 & 1 & 0 & 0 & 0 & 0 & 1 \\\hline
			1 & 1 & 1 & 1 & 0 & 1 & 1 \\\hline
		\end{tabular}
	\end{table*}
	Equivalent?: \hspace{1cm}yes\hspace{1cm}\fbox{no}
	\pagebreak 
	
	\question{$((p \land q) \Rightarrow (p \lor q)) \equiv 1$}{5}
	
	\textbf{Answer:}
	
	\begin{table*}[h!]
		\LARGE
		\begin{tabular}{c|c|c|c|c}
			$p$ & $q$ & $p \land q$ & $p \lor q$ & $(p \land q) \Rightarrow (p \lor q)$\\ \hline
			0 & 0 & 0 & 0 & 1 \\\hline
			0 & 1 & 0 & 1 & 1 \\\hline
			1 & 0 & 0 & 1 & 1 \\\hline
			1 & 1 & 1 & 1 & 1 \\\hline
		\end{tabular}
	\end{table*}
	Equivalent?: \hspace{1cm}\fbox{yes}\hspace{1cm}no
	
	
	
	\pagebreak
	
	
	\problem{Rules of inference}{24}%
	Examine the logical validity of the following argument forms with rules of inference. You should only use rules taught in this course. Show your work step by step, and label the rules that you use for each step. For Latex, you can try to use align* (or align) for aligned equations.  
	
	\bigskip
	
	\question{
		$$\begin{array}{rl}
			& (p \land \neg q) \Rightarrow r\\
			& p \lor q \\
			& q \Rightarrow 0 \\\hline
			\therefore & r
		\end{array}$$
	}{8}
	\textbf{Answer:}
	\begin{align*}
	    (1)\hspace{0.5cm}& (p \land \neg q) \Rightarrow r & \text{premise}\\
		(2)\hspace{0.5cm}& p \lor q & \text{premise} \\
		(3)\hspace{0.5cm}& q \Rightarrow 0 & \text{premise} \\
		(4)\hspace{0.5cm}& \lnot \lnot q \Rightarrow 0 & \text{Double negative (3)} \\
		(5)\hspace{0.5cm}& \lnot q & \text{Contradiction (4)} \\
		(6)\hspace{0.5cm}& p & \text{Elimination (2, 5)} \\
		(7)\hspace{0.5cm}& p \land q & \text{Conjunction (5, 6)} \\
		(8)\hspace{0.5cm}& r & \text{Modus Ponens (7)} \\
	\end{align*}
	\bigskip
	
	\question{
		$$\begin{array}{rl}
			& (p \land q)\\
			& p \Rightarrow s \\
			& r \lor \neg s \\\hline
			\therefore & r
		\end{array}$$
	}{8}
	\textbf{Answer:}
	\begin{align*}
	    (1)\hspace{0.5cm}& p \land q & \text{premise}\\
	    (2)\hspace{0.5cm}& p \Rightarrow s & \text{premise}\\
	    (3)\hspace{0.5cm}& r \lor \lnot s & \text{premise}\\
	    (4)\hspace{0.5cm}& p & \text{Specialization (1)}\\
	    (5)\hspace{0.5cm}& s & \text{Modus Ponens (2, 4)}\\
	    (6)\hspace{0.5cm}& \lnot \lnot s & \text{Double Negation (5)}\\
	    (7)\hspace{0.5cm}& r & \text{Elimination (3, 6)}\\
	\end{align*}
	\bigskip
	\bigskip
	\bigskip
	\bigskip
	\bigskip
	\bigskip
	\bigskip
	\bigskip
	\bigskip
	\bigskip
	
	\pagebreak
	\question{
		$$\begin{array}{rl}
			& q \lor \neg p\\
			& \neg s \Rightarrow \neg r \\
			& s \Rightarrow \neg q \\\hline
			\therefore & p \Rightarrow \neg r\\
		\end{array}$$
	}{8}
	\textbf{Answer:}
	\begin{align*}
	    (1)\hspace{0.5cm}& q \lor \neg p & \text{premise}\\
	    (2)\hspace{0.5cm}& \neg s \Rightarrow \neg r & \text{premise}\\
	    (3)\hspace{0.5cm}& s \Rightarrow \neg q & \text{premise}\\
	    (4)\hspace{0.5cm}& p \Rightarrow q & \text{Definition of implication (1)}\\
	    (5)\hspace{0.5cm}& \neg s \lor \neg q & \text{Definition of implication (3)}\\
	    (6)\hspace{0.5cm}& q \Rightarrow \neg s & \text{Definition of implication (5)}\\
	    (7)\hspace{0.5cm}& p \Rightarrow \neg s & \text{Transitivity (4, 6)}\\
	    (8)\hspace{0.5cm}& p \Rightarrow \neg r & \text{Transitivity (2, 7)}\\
	\end{align*}
	\bigskip
	\bigskip
	\bigskip
	\bigskip
	\bigskip
	\bigskip
	\bigskip
	\bigskip
	\bigskip
	\bigskip
	
	\pagebreak
	
	\problem{Validity}{20}%
	Are these inferences valid or invalid? Justify your answer using truth tables. Feel free to add columns to show your work, but the given columns must be present in your final answer. We can award partial credit if work is shown.

	\bigskip
	
	\question{
		$$\begin{array}{rl}
			& p\Leftrightarrow q\\\hline
			\therefore & p \Rightarrow q
		\end{array}$$
	}{10}
	\textbf{Answer:}
	\begin{table*}[h!]
		\LARGE
		\begin{tabular}{c|c|c|c|c}
			$p$ & $q$ & $q \Rightarrow p$ & $p\Leftrightarrow q$& $p \Rightarrow q$\\ \hline
			0 & 0 & 1 & 1 & 1 \\\hline
			0 & 1 & 0 & 0 & 1 \\\hline
			1 & 0 & 1 & 0 & 0 \\\hline
			1 & 1 & 1 & 1 & 1 \\\hline
		\end{tabular}
	\end{table*}
	Valid?: \hspace{1cm}\fbox{\textbf{yes}}\hspace{1cm}no
	
	\question{
		$$\begin{array}{rl}
			&p\Leftrightarrow q\\
			&q\Leftrightarrow p\\\hline
			\therefore & p \lor q
		\end{array}$$
	}{10}
	\textbf{Answer:}
	\begin{table*}[h!]
		\LARGE
		\begin{tabular}{c|c|c|c|c|c}
			$p$ & $q$ & \hspace{9cm} & $p\Leftrightarrow q$& $q\Leftrightarrow p$ & $p \lor q$\\ \hline
			0 & 0 & & 1 & 1 & 0  \\\hline
			0 & 1 & & 0 & 0 & 1 \\\hline
			1 & 0 & & 0 & 0 & 1 \\\hline
			1 & 1 & & 1 & 1 & 1 \\\hline
		\end{tabular}
	\end{table*}
	Valid?: \hspace{1cm}yes\hspace{1cm}\fbox{\textbf{no}}
	
	\pagebreak
	
	\problem{Can the following statements be true?}{30}%
	Determine whether these statements can be true (evaluate to $1$ for some input). If so, provide the set of values for the variables; if not, explain why. You can use laws of equivalence, or truth table.
	
	\bigskip
	
	\question{$((p \lor \neg q) \land (\neg p \lor q)) \land (\neg p \lor \neg q )$}{10}
	
	\textbf{Answer:}
	\begin{align*}
	    ((p \lor \neg q) \land (\neg p \lor q)) \land (\neg p \lor \neg q) &\equiv (p \lor \neg q) \land ((\neg p \lor q) \land (\neg p \lor \neg q)) & (\text{Associative}) \\
	    &\equiv (p \lor \neg q) \land (\neg p \lor (q \land \neg q)) & (\text{Distributive}) \\
	    &\equiv (p \lor \neg q) \land (\neg p \lor 0) & (\text{Negation}) \\
	    &\equiv (p \lor \neg q) \land \neg p & (\text{Identity}) \\
	    &\equiv (p \land \neg p) \lor (\neg q \land p) & (\text{Distributive}) \\
	    &\equiv 0 \lor (\neg q \land \neg p) & (\text{Negation}) \\
	    &\equiv (\neg q \land \neg p) & (\text{Identity})
	\end{align*}
	\fbox{$(p, q) = (0, 0)$}
	\bigskip
	\bigskip
	\bigskip
	\bigskip
	\bigskip
	\bigskip
	\bigskip
	
	\question{$(p \Rightarrow q) \land (p \Rightarrow \neg q) \land (\neg p \Rightarrow q) \land (\neg p \Rightarrow \neg q) $}{10}
	
	\textbf{Answer:}
	\begin{align*}
	    &(p \Rightarrow q) \land (p \Rightarrow \neg q) \land (\neg p \Rightarrow q) \land (\neg p \Rightarrow \neg q) \\
	    &\equiv (\lnot p \lor q) \land (\lnot p \lor \lnot q) \land (\lnot \lnot p \lor q) \land (\lnot \lnot p \lor \lnot q) & (\text{Definition of implication x2}) \\
	    &\equiv (\lnot p \lor q) \land (\lnot p \lor \lnot q) \land (p \lor q) \land (p \lor \lnot q) & (\text{Double negative x2}) \\
	    &\equiv (\lnot p \lor (q \land \lnot q)) \land (p \lor (q \land \lnot q)) & (\text{Distributive x2}) \\
	    &\equiv (\lnot p \lor 0) \land (p \lor 0) & (\text{Negation x2}) \\
	    &\equiv (\lnot p) \land (p) & (\text{Identity x2}) \\
	    &\equiv 0 & \text{Negation}
	\end{align*}
	No values of $p$ and $q$ can make the statement true because it leads to a contradiction.
	
	\pagebreak
	
	\question{$(p \Leftrightarrow q) \land (\neg p \Leftrightarrow q)$}{10}
	
	\textbf{Answer:}
	\begin{table*}[h!]
		\LARGE
		\begin{tabular}{c|c|c|c|c|c}
			$p$ & $q$ & $\lnot p$ & $p\Leftrightarrow q$ & $\lnot p \Leftrightarrow q$ & $(p\Leftrightarrow q) \land (\lnot p \Leftrightarrow q)$ \\ \hline
			0 & 0 & 1 & 1 & 0 & 0 \\\hline
			0 & 1 & 1 & 0 & 1 & 0 \\\hline
			1 & 0 & 0 & 0 & 1 & 0 \\\hline
			1 & 1 & 0 & 1 & 0 & 0 \\\hline
		\end{tabular}
	\end{table*}
	The statement can never be true.
	
	\pagebreak
	
	
	
\end{document}

%%% Local Variables:
%%% mode: latex
%%% TeX-master: t
%%% End:

