\documentclass[12pt, leqno]{article}
\usepackage[utf8]{inputenc}
\usepackage[margin=1in]{geometry}
\usepackage{amssymb}
\usepackage{amsmath}
\usepackage{parskip}

% Line before the therefore in proofs
\newcommand{\proofline}{\rule{0.75in}{0.5pt}}
% For set literals, wraps in {}
\newcommand{\set}[1]{\{#1\}}
% Powerset symbol
\newcommand{\powerset}{\mathcal{P}}
% Cardinality
\newcommand{\card}[1]{\lvert #1 \rvert}
\newcommand{\Z}{\mathbb Z}
\newcommand{\Q}{\mathbb Q}
\newcommand{\evens}{\Z^{\mathrm{even}}}
\newcommand{\odds}{\Z^{\mathrm{odd}}}
\newcommand{\Mod}[1]{\ (\mathrm{mod}\ #1)}

\title{Set Proofs}
\author{Yash Thakur}
\date{October 11, 2022}

\begin{document}
\counterwithout{equation}{section}

\maketitle

Useful facts/definitions:
\begin{itemize}
    \item Set equality: $A = B \Leftrightarrow (A \subseteq B) \land (B \subseteq A)$
    \item Subset: $A \subseteq B (\forall x, x \in A \Rightarrow x \in B)$
    \item Not subset: $A \not\subseteq B (\exists x, x \in A \land x \not\in B)$
    \item $x \in X \cup Y \Leftrightarrow x \in X \lor x \in Y$
    \item $x \in X \cap Y \Leftrightarrow x \in X \land x \in Y$
    \item Set difference: $x \in X - Y \Leftrightarrow x \in X \land x \not\in Y$
    \item $x \in X^c \Leftrightarrow x \not\in X$
    \item $(x, y) \in X \times Y \Leftrightarrow x \in X \land y \in Y$
\end{itemize}

\section*{Set Identities}

Commutative laws:
\begin{itemize}
    \item $A \cup B = B \cup A$
    \item $A \cap B = B \cap A$
\end{itemize}

Associative laws:
\begin{itemize}
    \item $(A \cup B) \cup C = A \cup (B \cup C)$
    \item $(A \cap B) \cap C = A \cap (B \cap C)$
\end{itemize}

Distributive laws:
\begin{itemize}
    \item $A \cup (B \cap C) = (A \cup B) \cap (A \cup C)$
    \item $A \cap (B \cup C) = (A \cap B) \cup (A \cap C)$
\end{itemize}

Identity laws:
\begin{itemize}
    \item $A \cup \{\} = A$
    \item $A \cap U = A$
\end{itemize}

Complement laws:
\begin{itemize}
    \item $A \cup A^c = U$
    \item $A \cap A^c = \{\}$
\end{itemize}

Double complement law: $(A^c)^c = A$

Idempotent laws:
\begin{itemize}
    \item $A \cup A = A$
    \item $A \cap A = A$
\end{itemize}

Universal bound laws:
\begin{itemize}
    \item $A \cup U = U$
    \item $A \cap \{\} = \{\}$
\end{itemize}

De Morgan's laws:
\begin{itemize}
    \item $(A \cup B)^c = A^c \cap B^c$
    \item $(A \cap B)^c = A^c \cup B^c$
\end{itemize}

Absorption laws:
\begin{itemize}
    \item $A \cup (A \cap B) = A$
    \item $A \cap (A \cup B) = A$
\end{itemize}

Complements of $U$ and $\{\}$:
\begin{itemize}
    \item $U^c = \{\}$
    \item $\{\}^c = U$
\end{itemize}

Set difference law: $A - B = A \cap B^c$

\section*{Proofs}

\subsection*{Example proof 1}

Prove $A - B = A \cap B^c$

To do this, must show $A - B \subseteq A \cap B^c$ \emph{and} $A \cap B^c \subseteq A - B$

\subsubsection*{Proving $A - B \subseteq A \cap B^c$}

(actual proof is like other proofs from before, just didn't bother with it here)
\begin{itemize}
    \item $x \in A - B \Rightarrow x \in A \land x \not\in B$
    \item $x \not\in B \Rightarrow x \in B^c$
    \item $x \in A \land x \in B^c \Rightarrow x \in A \cap B^c$
\end{itemize}

Therefore, $A - B \subseteq A \cap B^c$ because every element in $A - B$ is in $A \cap B^c$.

And the other part of the proof the same way.

\subsection{Example proof 2}

Disprove $A \cup B \subseteq A^c \cup B$

Just come up with a counterexample.

\end{document}
