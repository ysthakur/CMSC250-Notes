\documentclass[12pt]{article}
\usepackage[utf8]{inputenc}
\usepackage[margin=1in]{geometry}
\usepackage{amssymb}
\usepackage{amsmath}
\usepackage{parskip}

\newcommand{\proofline}{\rule{0.75in}{0.5pt}}

\begin{document}

\section*{Sets and Predicates}

\subsection*{Quantifiers}

Universal quantifier: $\forall$\\
Existential quantifier: $\exists$\\
Uniqueness quantifier: $\exists!$. There exists exactly one

Negation kinda uses DeMorgan's.

\subsection*{Predicates}

\textbf{Predicate:} A sentence that contains a finite number of variables and becomes a statement when specific values are substituted for the variables.

\textbf{Domain:} A set of all values that may be substituted in place of the variable.

\begin{enumerate}
    \item $\mathbb N$: $\{0, 1, 2, 3, ...\}$
    \item $\mathbb{Z}$: Integers
\end{enumerate}

\subsection*{Sets}

$U$ is the \textbf{universal set}, the set of all the stuff we're talking about.

$\varnothing$ is the empty set ($\{\}$).

Set builder notation: $\{ x \in S \mid P(x) \}$\\
Interval notation: $\{x \in \mathbb R \mid 1 \leq x \leq 2\}$

Subset: $A \subseteq B \Leftrightarrow (x \in A \Rightarrow x \in B)$\\
Equality: $A = B \Leftrightarrow ((A \subseteq B) \land (B \subseteq A))$\\
Proper subset: $A \subset B \Leftrightarrow ((x \in A \Rightarrow x \in B) \land (\exists y \in B, y \not\in A))$

Union: $A \cup B: \{x \in U | x \in A \lor x \in B\}$\\
Intersection: $\cap$

\textbf{Disjoint sets} are two sets that have no elements in common.

Set complement: $A' = \{x \in U \mid x \not\in A \}$

Set difference (a.k.a. relative complement): $B - A = \{x \mid (x \in B) \land (x \not\in A)\}$ (sometimes written $B\backslash A$)

\end{document}
