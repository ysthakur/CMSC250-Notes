\documentclass[leqno]{article}

\setlength{\oddsidemargin}{0in}
\setlength{\textwidth}{6in}
\setlength{\topmargin}{-0.1in}
\setlength{\textheight}{8.2in}

%%%%%%%%%%%%%  IMPORT MACRO FILES AS NEEDED %%%%%%%%%%%
\usepackage{amsgen,amsmath,amstext,amsbsy,amsopn,amssymb,amsthm,stackengine}
\usepackage{array, nicefrac, mathtools}
\usepackage{verbatim}
\usepackage{hyperref}
\usepackage{float,relsize,setspace,enumitem,pbox,cleveref,multicol,multirow}
\usepackage{multido}
\usepackage{bbding} % Has a checkmark symbol reachable through \Checkmark
\usepackage{tikz,mdframed}
\usepackage{listings}
% \usepackage{circuitikz}

% Theorems, definitions, equations, lemmas
\newtheorem{thm}{Theorem}[section]
\newtheorem{prop}[thm]{Proposition}
\newtheorem{lem}[thm]{Lemma}
\newtheorem{cor}[thm]{Corollary}
\newtheorem{defn}{Definition}
\newtheorem{rem}[thm]{Remark}
\numberwithin{equation}{section}
\newtheorem*{defn*}{Definition} % Theorem environments with no numbering
\newtheorem*{prop*}{Proposition}
\newtheorem*{thm*}{Theorem}
\theoremstyle{definition}
\newtheorem*{fact}{Fact}

% For negation and quantifiers in Discrete Math
\newcommand{\shortsim}{\raise.17ex\hbox{$\scriptstyle \sim$}}
\renewcommand{\neg}{\shortsim}
\renewcommand{\nexists}{\neg(\exists}
\newcommand{\nequiv}{\ensuremath{\not\equiv}}

\newcommand{\myline}[1]{\underline{\hspace{#1}}}
\newcommand*\emptycirc[1][1ex]{\tikz\draw (0,0) circle (#1);} 
\newcounter{parts}
\newcounter{problems}[parts]
\newcounter{questions}[problems]
\newcounter{subquestions}[questions]
\newcommand{\hwpart}[1]{
  \stepcounter{parts}
  \noindent\makebox[\textwidth]{\LARGE \bf Part \arabic{parts} - #1}
  \\
}
\newcommand{\problem}[2]{\stepcounter{problems}
  {\Large \bf \noindent Problem \arabic{problems}: #1 \marginpar{[Total #2 pts]} \\[0.3cm]}}
\newcommand{\question}[2]{\stepcounter{questions}
  {\large (\alph{questions}) #1 \marginpar{[#2 pts]} \\[.3cm]}}
\newcommand{\subquestion}[2]{\stepcounter{subquestions}
  {\hspace{10pt}\emph{(\roman{subquestions}) #1 \marginpar{[#2 pts]} }\\[.3cm]}}

% Solution formatting
\newcommand{\solution}[1]{{\color{red}{#1}}}
% Some standard centering and italicization of text.
\newcommand{\frontrowcenter}[1]{\begin{center}{\em \Large  #1  }\end{center}}

% A blank page
\newcommand{\blankpage}{
\clearpage
\vspace*{\fill}
\begin{minipage}{\textwidth}
  \Large \textbf{THIS PAGE INTENTIONALLY LEFT BLANK}\\
\end{minipage}
\vfill % equivalent to \vspace{\fill}
\clearpage
}

\newcommand{\answerspace}[1]{
  \begin{center}
    \textbf{BEGIN YOUR ANSWER BELOW THIS LINE} \\ \hrulefill \vspace{#1} \\ \hrulefill
  \end{center}
}

\newcommand{\answerspacefullpage}{
  \begin{center}
    \textbf{BEGIN YOUR ANSWER BELOW THIS LINE} \\ \hrulefill \pagebreak
  \end{center}
}

\newcommand{\additionalanswerspace}[1]{
  \begin{center}
    \textbf{CONTINUE YOUR ANSWER BELOW THIS LINE } \\ \hrulefill \vspace{#1} \\ \hrulefill
  \end{center}
}

\newcommand{\additionalanswerspacefullpage}{
  \begin{center}
    \textbf{CONTINUE YOUR ANSWER BELOW THIS LINE} \\ \hrulefill \pagebreak
  \end{center}
}

\newcommand{\freespace}[1]{
  \begin{center}
    \large \textbf{SCRAP SPACE BELOW} \\
    \hrulefill
    \pagebreak
  \end{center}
}

% Centered line
\newcommand{\mycenterline}[1]{
  \begin{center}
    \myline{#1}
  \end{center}
}

% Space for T/F:
\newcommand{\tfline}{\myline{.5cm}}

% For quick parenthesized and italicized point annotation.
\newcommand{\pts}[1]{{\em (#1 pts)}}
\newcommand{\onept}{{\em (1 pt)}}

% \item environments coupled with a line at the end, for students to write T and F in.
\newcommand{\tfitem}[1]{\item #1 \null\hfill \framebox(25,25){} \\ \hdashrule{0.95\textwidth}{1pt}{2pt}}
\newcommand{\setitem}[1]{\tfitem{$\curlybraces{#1}$} }
\newcommand{\lineitem}[2]{\item #1 \null \hfill \myline{#2}}

% Some circles and squares for students to fill in.
\newcommand{\whitecircle}[1]{\tikz[baseline=-0.5ex]\draw[black, radius=#1] (0,0) circle ;}
\newcommand{\whitesquare}[1]{\tikz\draw[black] (0,0) rectangl#1, #1) ;}

% Emphasis
\newcommand{\F}{$\mathbf{F}$}
\newcommand{\T}{$\mathbf{T}$}
\newcommand{\False}{\textbf{False}}
\newcommand{\false}{\textbf{false}}
\newcommand{\True}{\textbf{True}}
\newcommand{\true}{\textbf{true}}
\newcommand{\makered}[1]{\textcolor{red}{#1}}
\newcommand{\Rbbst}{\textcolor{red}{Red}-black tree}
\newcommand{\rbbst}{\textcolor{red}{red}-black tree}

\newcommand{\homeworkdata}[4]{
  \begin{mdframed}[linewidth=1pt]
    \noindent\makebox[\textwidth]{\LARGE \bf #1, #2 }
    \\\\
    \noindent\makebox[\textwidth]{\Large \bf  Homework \##3 }
    \\\\
    \noindent\makebox[\textwidth]{\large \bf  Due: #4}
    \\\\
    \noindent\makebox[\textwidth]{\large \bf Homework will not be accepted late}
  \end{mdframed}
  \vspace{40pt}
}

\usepackage{circuitikz}

\setlength{\parindent}{0em}
\setlength{\itemindent}{.5in}

\newcommand{\poneanswer}{%
}
\newcommand{\ptwoanswer}{%
}
\newcommand{\pthreeanswer}{%
}
\newcommand{\pfouranswer}{%
}
\newcommand{\pfiveanswer}{%
}
\newcommand{\psixanswer}{%
}
% \include{solutions}

%%%%%%%%%%%%%%%%%%%%%%%%%%%%%%%%%%%%%%%%%%%%%
%
%  STUDENTS - Your homework begins here.
%
%%%%%%%%%%%%%%%%%%%%%%%%%%%%%%%%%%%%%%%%%%%%%

\begin{document}
\pagestyle{empty}

\homeworkdata{CMSC 250}{Fall 2022}{11}{Sunday 11 Dec.\ 11:59pm}

{\Large \bf
  \begin{center}
  IMPORTANT
  \end{center}

  You can write your answers on any paper, either this paper
  or blank paper, or write your answer in Latex (template of this homework can be downloaded through ELMS).
  
  When you upload your document to Gradescope, make sure you tag your questions.

  \begin{center}
    YOU WILL NEED TO TAG YOUR PROBLEMS!!!
  \end{center}

  Problems which are not correctly found will not be graded, this is a zero-tolerance policy. 

  \begin{center}
    IF YOU ARE WORRIED...
  \end{center}

  If you have concerns about tagging your problems,
  We strongly suggest you drop by office hours and do it with a TA present so they can help you through the process,
  just to see how it works. In addition, Gradescope has a tutorial: \url{https://help.gradescope.com/article/ccbpppziu9-student-submit-work#submitting_a_pdf}


}

\pagebreak

\problem{Bayes}{20}

\question{A factory produces a batch of toy vehicles, $3/8$ of which are cars and the rest of which are trucks. Trucks are defective 13\% of the time and cars are defective 4\% of the time. The quality inspector picks a toy vehicle at random from the batch and finds out that it is defective. What is the probability that the vehicle was a truck?
}{10}

$\displaystyle P(\text{truck} \mid \text{defective}) = \frac{P(\text{defective} \mid \text{truck})P(\text{truck})}{P(\text{defective})} = \frac{0.13 \cdot \frac{5}{8}}{0.13 \cdot \frac{5}{8} + 0.04 \cdot \frac{3}{8}} \approx 0.84$

\bigskip

\question{One night, a taxi hit another car and ran. The city has two taxi companies, one which has green taxis and another which has blue. The green taxi company owns 74\% of the cities' taxis and the blue taxi company owns the remaining 26\% of the taxis. A witness claimed that the taxi involved in the incident is blue. A test performed by court shows that the witness has 92\% accuracy when discriminating color in an emergency situation. What is the probability that the taxi involved in the incident is blue?}{10}

$P(\text{right})$: probability that witness was right\\
$P(\text{blue})$: probability that taxi was blue\\
$P(\text{green})$: probability that taxi was green\\
$P(\text{saw blue})$: probability that witness saw blue taxi

\bigskip

$P(\text{blue} \cap \text{right}) = 0.26 \cdot 0.92 = 0.2392$
\begin{flalign*}
    P(\text{blue} \mid \text{saw blue}) &= \frac{P(\text{blue} \cap \text{saw blue})}{P(\text{saw blue})} &\\
    &= \frac{P(\text{blue} \cap \text{right})}{P(\text{blue} \cap \text{right}) + P(\text{green} \cap \text{wrong})}\\
    &= \frac{P(\text{blue} \cap \text{right})}{P(\text{blue} \cap \text{right}) + P(\text{green})P(\text{wrong})}\\
    &= \frac{0.2392}{0.2392 + 0.74 \cdot 0.08}\\
    &\approx 0.80
\end{flalign*}

There's approximately an 80\% chance that the taxi involved in the incident is blue.

\pagebreak

\problem{Algorithm}{80}

For this question, \textbf{do not} express answers in Big O notation; present your final anwer in polynomial form (for example, $3n^2 + 2n -5$). Remember to count additions and assignments in for loops.

\question{What is the number of additions used in this segment of a pseudocode? 
}{20}

\begin{lstlisting}
    t := 0
    for i := 1 to n
        for j := 1 to n
            t := t + i + j
\end{lstlisting}
$n$ is a positive number.

There are $n-1$ additions to increment $i$.\\
There are $n(n-1)$ additions to increment $j$.\\
$t = t + i + j$ has 2 additions, so the inner loop has $2(1 + n)$ additions, so the outer loop has $2(1+n)^2$ additions.

Total additions: $n-1 + n(n-1) + 2(1+n)^2 = n - 1 + n^2 - n + 2 + 4n + 2n^2 = \fbox{1 + 4n + 3n^2}$

\pagebreak


\begin{lstlisting}

    t := a list of n numbers
    max := t[0]
    for x in t:
        if x > max:
            max := x
\end{lstlisting}
$n$ is a positive number.

\question{For the piece of code above, what is the number of assignments in the best case?}{15}

In the best case, t[0] will be the max of t, so there will only be 2 assignments, one for initializing t and one for initializing max to t[0].

\bigskip
\bigskip
\bigskip

\question{For the piece of code above, what is the number of assignments in the worst case?}{10}

In the worst case, t is strictly increasing, so there will be one assignment for each element in t (not including t[0]), which is $n-1$ assignments. So counting the initialization of t and max, there are $n-1+2=n+1$ assignments in all.

\bigskip
\bigskip
\bigskip

\question{For the piece of code above, what is the number of assignments in the average case if $P(t[i] > max) = 1 - \frac{1}{(i+1)^2}$? You can leave your answer in the form of a summation.}{15}

There will always be the 2 assignments at the top for t and max.\\
$P(t[i] > max)$ is the probability that there will be an assignment at index $i$. Adding up those probabilities gives you the expected number of assignments.\\
$\displaystyle 2 + \sum_{i = 1}^{n-1} \left(1 - \frac{1}{(i+1)^2}\right) = 1 + n - \sum_{i = 1}^{n-1} \frac{1}{(i+1)^2}$


\bigskip
\bigskip
\bigskip

\end{document}

%%% Local Variables:
%%% mode: latex
%%% TeX-master: t
%%% End:

