\documentclass[12pt, leqno]{article}
\usepackage[utf8]{inputenc}
\usepackage[margin=1in]{geometry}
\usepackage{amssymb}
\usepackage{amsmath}
\usepackage{parskip}
\usepackage{ulem}

% Line before the therefore in proofs
\newcommand{\proofline}{\rule{0.75in}{0.5pt}}
% For set literals, wraps in {}
\newcommand{\set}[1]{\{#1\}}
% Powerset symbol
\newcommand{\powerset}{\mathcal{P}}
% Cardinality
\newcommand{\card}[1]{\lvert #1 \rvert}
\newcommand{\Z}{\mathbb Z}
\newcommand{\N}{\mathbb N}
\newcommand{\Q}{\mathbb Q}
\newcommand{\R}{\mathbb R}
\newcommand{\evens}{\Z^{\mathrm{even}}}
\newcommand{\odds}{\Z^{\mathrm{odd}}}
\newcommand{\Mod}[1]{\ (\mathrm{mod}\ #1)}

% permutations
\newcommand{\perm}[2]{{}_{#1}\mathrm{P}_{#2}}
% combinations
\newcommand{\comb}[2]{{}_{#1}\mathrm{C}_{#2}}

\title{Practice for Final}
\author{Yash Thakur}
\date{}

\begin{document}

\maketitle

\section{Proofs}

\subsection{Direct Proofs}

\begin{enumerate}
    \item Prove that $\forall n \in \N, 4 \mid (2n^3 + 6n^2 + 4n)$

    We can prove this by cases. $n$ is either even or odd.

    \textbf{Case 1:} $n$ is even.\\
    Since $n$ is even, there exists a $k \in \Z$ such that $n = 2k$.
    \begin{flalign*}
        2n^3 + 6n^2 + 4n &= 2(2k)^3 + 6(2k)^2 + 4(2k) & \text{substitution} & \\
        &= 4(4k^3) + 4(6k^2) + 4(2k) & \text{arithmetic} & \\
        &= 4(4k^3 + 6k^2 + 2k) & \text{arithmetic}
    \end{flalign*}
    $4k^3 + 6k^2 + 2k \in \Z$ since the integers are closed under multiplication and addition.\\
    Therefore, $4 \mid 4(4k^3 + 6k^2 + 2k) = 2n^3 + 6n^2 + 4n$

    \textbf{Case 2:} $n$ is odd.\\
    Since $n$ is odd, there exists a $k \in \Z$ such that $n = 2k + 1$.
    \begin{flalign*}
        2n^3 + 6n^2 + 4n &= 2(2k+1)^3 + 6(2k+1)^2 + 4(2k+1) & \text{substitution} & \\
        &= (2k+1)(2(2k+1)^2 + 6(2k+1) + 4) & \text{arithmetic} & \\
        &= (2k+1)(2(4k^2+2k+1) + 12k+6 + 4) & \text{arithmetic} & \\
        &= (2k+1)(8k^2+4k+2 + 12k+6 + 4) & \text{arithmetic} & \\
        &= (2k+1)(8k^2+16k+12) & \text{arithmetic} & \\
        &= 4(2k+1)(2k^2+4k+3) & \text{arithmetic} & \\
    \end{flalign*}
    $(2k+1)(2k^2+4k+3) \in \Z$ since the integers are closed under multiplication and addition.\\
    Therefore, $4 \mid 4(2k+1)(2k^2+4k+3) = 2n^3 + 6n^2 + 4n$

    In both cases, $4 \mid 2n^3 + 6n^2 + 4n$. Therefore, we have proven that $\forall n \in \N, 4 \mid 2n^3 + 6n^2 + 4n$

    \item Prove that $\forall m, n \in \N, 6 \mid m \land 4 \mid n \Rightarrow 2 \mid (5m - 7n)$

    This can be proven using a direct proof

    From the premises, we have $m, n \in \N$ such that $6 \mid m \land 4 \mid n$\\
    So $6 \mid m$ and $4 \mid n$ using specialization\\
    Since $6 \mid m$, there exists a $k \in \Z$ such that $m = 6k$\\
    Since $4 \mid n$, there exists a $j \in \Z$ such that $n = 4k$

    $5m - 7n = 5(6k) - 7(4n)$ using substitution\\
    $5m - 7n = 2(15k) - 2(14n)$\\
    $5m - 7n = 2(15k - 14n)$\\
    Since the integers are closed under addition and multiplication, $15k - 14n \in \Z$\\
    Therefore, $2 \mid 2(15k - 14n) = 5m - 7n$

    Thus, we have proved that $\forall m, n \in \N, 6 \mid m \land 4 \mid n \Rightarrow 2 \mid (5m - 7n)$
\end{enumerate}

\subsection{Proof by Contrapositive}

\begin{enumerate}
    \item Prove that for all sets $A, B, C$ which are a subset of a universal set $U$ that $(A - C) \cap B \neq \varnothing \Rightarrow A \cap B \not\subseteq C$

    This can be done with a proof by contrapositive. We can prove instead that $A \cap B \subseteq C \Rightarrow (A - C) \cap B = \varnothing$
    \begin{flalign}
        & A \cap B \subseteq C & \text{Premise} & \\
        & \forall x \in A \cap B, x \in C & \text{Definition of $\subseteq$} \\
        & \forall x \in A - C, x \in A \land x \not\in C & \text{Definition of $-$} \\
        & \forall x \in A - C, x \in A & \text{Specialization (3)} \\
        & \forall x \in A - C, x \not\in C & \text{Specialization (3)} \\
        & \forall x \in (A - C) \cap B, x \in (A - C) \land x \in B & \text{Definition of $\cap$} \\
        & \forall x \in (A - C) \cap B, x \in (A - C) & \text{Specialization (5)} \\
        & \forall x \in (A - C) \cap B, x \in B & \text{Specialization (6)} \\
        & \forall x \in (A - C) \cap B, x \in A & \text{Modus Ponens (7, 4)} \\
        & \forall x \in (A - C) \cap B, x \in A \land x \in B & \text{Conjunction (8, 9)} \\
        & \forall x \in (A - C) \cap B, x \in A \cap B & \text{Definition of $\cap$ (10)} \\
        & \forall x \in (A - C) \cap B, x \in C & \text{Modus Ponens (11, 1)} \\
        & \forall x \in (A - C) \cap B, x \not\in C & \text{Modus Ponens (7, 5)} \\
        & \forall x \in (A - C) \cap B, x \in C \land x \not\in C & \text{Conjunction (12, 13)} \\
        & \neg \exists x, x \in (A - C) \cap B & \text{Contradiction (14)}
    \end{flalign}

    Since there are no elements in $(A - C) \cap B$, we know that $(A - C) \cap B = \varnothing$.

    Thus, we have proven that $(A - C) \cap B \neq \varnothing \Rightarrow A \cap B \not\subseteq C$.

    \hline

    Direct proof below not right approach

    From the premise, we know that $(A - C) \cap B \neq \varnothing$.\\
    Because of the definition of set equality, we know that $(A - C) \cap B \not\subseteq \varnothing \lor \varnothing \not\subseteq (A - C) \cap B$.\\

    We can now split into two cases based on that disjunction.

    \textbf{Case 1:} $\varnothing \not\subseteq (A - C) \cap B$\\
    Since the empty set is a subset of all sets, $\varnothing \subseteq (A - C) \cap B$.\\
    However, this contradicts with the assumption for this case that $\varnothing \not\subseteq (A - C) \cap B$.\\
    Therefore, this case is not possible.

    \textbf{Case 2:} $(A - C) \cap B \not\subseteq \varnothing$
    \begin{flalign}
        & (A - C) \cap B \not\subseteq \varnothing & \text{Case assumption} & \\
        & \neg \forall x \in (A - C) \cap B, x \in \varnothing & \text{Definition of $\subseteq$ (1)} \\
        & \exists x \in (A - C) \cap B, x \not\in \varnothing & \text{De Morgan's (2)} \\
        & x \in (A - C) \land x \in B & \text{Definition of $\cap$ (3)} \\
        & x \in (A - C) & \text{Specialization (4)} \\
        & x \in B & \text{Specialization (4)} \\
        & x \in A \land x \not\in C & \text{Definition of $-$ (5)} \\
        & x \in A & \text{Specialization (7)} \\
        & x \not\in C & \text{Specialization (7)} \\
        & (x \in A) \land (x \in B) & \text{Conjunction (6, 8)} \\
        & x \in A \cap B & \text{Definition of $\cap$ (10)} \\
        & \exists y \in A \cap B, y \not\in C & \text{(9, 11)} \\
        & \neg \forall y \in A \cap B, y \in C & \text{De Morgan's (12)} \\
        & \neg (A \cap B \subseteq C) & \text{Definition of $\subseteq$ (13)}
    \end{flalign}

    In both cases, we have proven that $A \cap B \not\subseteq C$

    Therefore, we have proven $(A - C) \cap B \neq \varnothing \Rightarrow A \cap B \not\subseteq C$

    \item Let $x, y$ be two integers. Suppose $x^2(y^2 - 2y)$ is odd. Prove that $x$ and $y$ are odd.

    This can be done using a proof by contrapositive. We can instead prove that if $x$ and $y$ are even, then $x^2(y^2 - 2y)$ is even.

    Since $x$ is even, there exists a $k \in \Z$ such that $x = 2k$.\\
    $x^2(y^2 - 2y) = (2k)^2(y^2 - 2y)$\\
    $x^2(y^2 - 2y) = 2(2k^2)(y^2 - 2y)$\\
    Since the integers are closed under addition and multiplication, $(2k^2)(y^2 - 2y) \in \Z$\\
    So $2 \mid 2(2k^2)(y^2 - 2y) = x^2(y^2 - 2y)$

    Thus, we have proven that if $x^2(y^2 - 2y)$ is odd, then $x$ and $y$ are odd.
\end{enumerate}

\subsection{Proof by Contradiction}

\begin{enumerate}
    \item Prove $\forall x, y \in \Z^+, x^2 - y^2 \neq 1$

    This can be done using a proof by contradiction.

    Assume that $\exists x, y \in \Z^+, x^2 - y^2 = 1$\\
    So by factoring, $(x + y)(x - y) = 1$

    We can split on two cases now, $x = y$ or $x \neq y$

    \textbf{Case 1:} $x = y$\\
    $(x + x)(x - x) = 1$\\
    $(x + x)0 = 1$\\
    $0 = 1$\\
    This is a contradiction, so it's not possible that $x = y$

    \textbf{Case 2:} $x \neq y$\\
    $(x + y) = \frac 1{x-y}$\\
    Since $x + y \in \Z$, we also have that $\frac 1{x-y} \in \Z$\\
    For $\frac 1{x-y}$ to be an integer, 1 must be divisible by $x-y$\\
    This means that $x-y$ is either -1 or 1\\
    This means that $\frac 1{x-y}$ is either -1 or 1\\
    So $x+y$ is either -1 or 1\\
    But this cannot be true, since both $x$ and $y$ are positive, so their sum must be at least 2.\\
    This is a contradiction, so it's not possible that $x \neq y$

    Since both cases result in contradictions, it's not possible that $x^2-y^2=1$

    Thus, we have proven that $\forall x, y \in \Z^+, x^2 - y^2 \neq 1$

    \item Prove that $\sqrt[3]{2}$ is irrational.

    \textbf{Lemma:} For any integer $x$, if $x^3$ is even, then $x$ is even.

    This can be done using a proof by contrapositive. We can prove instead that if $x$ is odd, then $x^3$ is odd.

    Since $x$ is odd, there exists a $k \in \Z$ such that $x = 2k+1$\\
    Substituting that in, $x^3 = (2k+1)^3 = 8k^3 + 12k^2 + 6k + 1 = 2(4k^3 + 6k^2 + 3k) + 1$\\
    Since the integers are closed under addition and multiplication, $4k^3 + 6k^2 + 3k \in \Z$\\
    So $2(4k^3 + 6k^2 + 3k) + 1$ is odd, which means $x^3$ is odd.

    Thus, we have proven that if $x^3$ is even, then $x$ is even.

    \textbf{Proof:}

    This can be done using a proof by contradiction.

    Assume that $\sqrt[3]{2}$ is rational.

    Since $\sqrt[3]{2}$ is rational, $\exists p, q \in \Z, \sqrt[3]{2} = \frac p q, q \neq 0$. WLOG, let $p$ and $q$ share no common factors so that $\frac p q$ is simplified.\\
    So $2 = \frac{p^3}{q^3}$\\
    $p^3 = 2q^3$\\
    $q^3 \in \Z$ since $q \in \Z$ and the integers are closed under multiplication\\
    So $2 \mid 2q^3$, which means $2 \mid p^3$\\
    Since $p^3$ is even, $p$ is also even according to the lemma above.\\
    So there exists a $k \in \Z$ such that $2 \mid p$\\
    $2q^3 = p^3 \Rightarrow 2q^3 = 8k^3 \Rightarrow q^3 = 2(2k^3)$\\
    $2k^3 \in \Z$ since the integers are closed under multiplication\\
    So $2 \mid 2(2k^3) = q^3$\\
    Since $2 \mid q^3$, by the lemma, $2 \mid q$\\
    Since $q$ is even, there exists a $j \in \Z$ such that $q = 2j$\\
    So $p$ and $q$ have a common factor (2).\\
    This is a contradiction, as we said at the beginning that they don't have a common factor

    Therefore, $\sqrt[3]{2}$ is not rational.
\end{enumerate}

\section{Relations}

\begin{enumerate}
    \item Reflexive, symmetric, transitive
    \item Reflexive, symmetric, transitive
    \item Reflexive, not symmetric, transitive
\end{enumerate}

\section{Functions}

\begin{enumerate}
    \item \sout{None} Surjective, since it's naturals, not reals
    \item \sout{Bijective} Injective, since it's naturals, not integers
    \item Bijective
\end{enumerate}

\end{document}
