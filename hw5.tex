\documentclass[leqno]{article}

\setlength{\oddsidemargin}{0in}
\setlength{\textwidth}{6in}
\setlength{\topmargin}{-0.1in}
\setlength{\textheight}{8.2in}

%%%%%%%%%%%%%  IMPORT MACRO FILES AS NEEDED %%%%%%%%%%%
\usepackage{amsgen,amsmath,amstext,amsbsy,amsopn,amssymb,amsthm,stackengine}
\usepackage{array, nicefrac, mathtools}
\usepackage{verbatim}
\usepackage{hyperref}
\usepackage{float,relsize,setspace,enumitem,pbox,cleveref,multicol,multirow}
\usepackage{multido}
\usepackage{bbding} % Has a checkmark symbol reachable through \Checkmark
\usepackage{tikz,mdframed}
% \usepackage{circuitikz}

% Theorems, definitions, equations, lemmas
\newtheorem{thm}{Theorem}[section]
\newtheorem{prop}[thm]{Proposition}
\newtheorem{lem}[thm]{Lemma}
\newtheorem{cor}[thm]{Corollary}
\newtheorem{defn}{Definition}
\newtheorem{rem}[thm]{Remark}
\numberwithin{equation}{section}
\newtheorem*{defn*}{Definition} % Theorem environments with no numbering
\newtheorem*{prop*}{Proposition}
\newtheorem*{thm*}{Theorem}
\theoremstyle{definition}
\newtheorem*{fact}{Fact}

% For negation and quantifiers in Discrete Math
\newcommand{\shortsim}{\raise.17ex\hbox{$\scriptstyle \sim$}}
\renewcommand{\neg}{\shortsim}
\renewcommand{\nexists}{\neg(\exists}
\newcommand{\nequiv}{\ensuremath{\not\equiv}}

\newcommand{\myline}[1]{\underline{\hspace{#1}}}
\newcommand*\emptycirc[1][1ex]{\tikz\draw (0,0) circle (#1);} 
\newcounter{parts}
\newcounter{problems}[parts]
\newcounter{questions}[problems]
\newcounter{subquestions}[questions]
\newcommand{\hwpart}[1]{
  \stepcounter{parts}
  \noindent\makebox[\textwidth]{\LARGE \bf Part \arabic{parts} - #1}
  \\
}
\newcommand{\problem}[2]{\stepcounter{problems}
  {\Large \bf \noindent Problem \arabic{problems}: #1 \marginpar{[Total #2 pts]} \\[0.3cm]}}
\newcommand{\question}[2]{\stepcounter{questions}
  {\large (\alph{questions}) #1 \marginpar{[#2 pts]} \\[.3cm]}}
\newcommand{\subquestion}[2]{\stepcounter{subquestions}
  {\hspace{10pt}\emph{(\roman{subquestions}) #1 \marginpar{[#2 pts]} }\\[.3cm]}}

% Solution formatting
\newcommand{\solution}[1]{{\color{red}{#1}}}
% Some standard centering and italicization of text.
\newcommand{\frontrowcenter}[1]{\begin{center}{\em \Large  #1  }\end{center}}

% A blank page
\newcommand{\blankpage}{
\clearpage
\vspace*{\fill}
\begin{minipage}{\textwidth}
  \Large \textbf{THIS PAGE INTENTIONALLY LEFT BLANK}\\
\end{minipage}
\vfill % equivalent to \vspace{\fill}
\clearpage
}

\newcommand{\answerspace}[1]{
  \begin{center}
    \textbf{BEGIN YOUR ANSWER BELOW THIS LINE} \\ \hrulefill \vspace{#1} \\ \hrulefill
  \end{center}
}

\newcommand{\answerspacefullpage}{
  \begin{center}
    \textbf{BEGIN YOUR ANSWER BELOW THIS LINE} \\ \hrulefill \pagebreak
  \end{center}
}

\newcommand{\additionalanswerspace}[1]{
  \begin{center}
    \textbf{CONTINUE YOUR ANSWER BELOW THIS LINE } \\ \hrulefill \vspace{#1} \\ \hrulefill
  \end{center}
}

\newcommand{\additionalanswerspacefullpage}{
  \begin{center}
    \textbf{CONTINUE YOUR ANSWER BELOW THIS LINE} \\ \hrulefill \pagebreak
  \end{center}
}

\newcommand{\freespace}[1]{
  \begin{center}
    \large \textbf{SCRAP SPACE BELOW} \\
    \hrulefill
    \pagebreak
  \end{center}
}

% Centered line
\newcommand{\mycenterline}[1]{
  \begin{center}
    \myline{#1}
  \end{center}
}

% Space for T/F:
\newcommand{\tfline}{\myline{.5cm}}

% For quick parenthesized and italicized point annotation.
\newcommand{\pts}[1]{{\em (#1 pts)}}
\newcommand{\onept}{{\em (1 pt)}}

% \item environments coupled with a line at the end, for students to write T and F in.
\newcommand{\tfitem}[1]{\item #1 \null\hfill \framebox(25,25){} \\ \hdashrule{0.95\textwidth}{1pt}{2pt}}
\newcommand{\setitem}[1]{\tfitem{$\curlybraces{#1}$} }
\newcommand{\lineitem}[2]{\item #1 \null \hfill \myline{#2}}

% Some circles and squares for students to fill in.
\newcommand{\whitecircle}[1]{\tikz[baseline=-0.5ex]\draw[black, radius=#1] (0,0) circle ;}
\newcommand{\whitesquare}[1]{\tikz\draw[black] (0,0) rectangl#1, #1) ;}

% Emphasis
\newcommand{\F}{$\mathbf{F}$}
\newcommand{\T}{$\mathbf{T}$}
\newcommand{\False}{\textbf{False}}
\newcommand{\false}{\textbf{false}}
\newcommand{\True}{\textbf{True}}
\newcommand{\true}{\textbf{true}}
\newcommand{\makered}[1]{\textcolor{red}{#1}}
\newcommand{\Rbbst}{\textcolor{red}{Red}-black tree}
\newcommand{\rbbst}{\textcolor{red}{red}-black tree}

\newcommand{\homeworkdata}[4]{
  \begin{mdframed}[linewidth=1pt]
    \noindent\makebox[\textwidth]{\LARGE \bf #1, #2 }
    \\\\
    \noindent\makebox[\textwidth]{\Large \bf  Homework \##3 }
    \\\\
    \noindent\makebox[\textwidth]{\large \bf  Due: #4}
    \\\\
    \noindent\makebox[\textwidth]{\large \bf Homework will not be accepted late}
  \end{mdframed}
  \vspace{40pt}
}

\usepackage{circuitikz}

\setlength{\parindent}{0em}
\setlength{\itemindent}{.5in}

\newcommand{\poneanswer}{%
}
\newcommand{\ptwoanswer}{%
}
\newcommand{\pthreeanswer}{%
}
\newcommand{\pfouranswer}{%
}
\newcommand{\pfiveanswer}{%
}
\newcommand{\psixanswer}{%
}
% \include{solutions}

%%%%%%%%%%%%%%%%%%%%%%%%%%%%%%%%%%%%%%%%%%%%%
%
%  STUDENTS - Your homework begins here.
%
%%%%%%%%%%%%%%%%%%%%%%%%%%%%%%%%%%%%%%%%%%%%%

\begin{document}
\pagestyle{empty}

\homeworkdata{CMSC 250}{Fall 2022}{5}{Sunday 16 Oct.\ 11:59pm}

{\Large \bf
  \begin{center}
  IMPORTANT
  \end{center}

  You can write your answers on any paper, either this paper
  or blank paper, or write your answer in Latex (template of this homework can be downloaded through ELMS).
  
  When you upload your document to Gradescope, make sure you tag your questions.

  \begin{center}
    YOU WILL NEED TO TAG YOUR PROBLEMS!!!
  \end{center}

  Problems which are not correctly found will not be graded, this is a zero-tolerance policy. 

  \begin{center}
    IF YOU ARE WORRIED...
  \end{center}

  If you have concerns about tagging your problems,
  We strongly suggest you drop by office hours and do it with a TA present so they can help you through the process,
  just to see how it works. In addition, Gradescope has a tutorial: \url{https://help.gradescope.com/article/ccbpppziu9-student-submit-work#submitting_a_pdf}


}

\pagebreak
\counterwithout{equation}{section}
\problem{Set, Power set, partition, Cartesian Product}{25}
\question{What is the cardinality of the following sets?}{5}
\begin{itemize}
    \item $\{(x,y) \in \{15\}\times \mathbb{Z}| (y|x) \}$ \\ \textbf{Answer:} 8
    \item $\{\{\{4,5\},6\}\}$ \\ \textbf{Answer:} 1
    \item $\emptyset$ \\ \textbf{Answer:} 0
    \item $\mathbb{Z}$ \\ \textbf{Answer:} $\aleph_0$
    \item $\mathbb{R}\cap [1, 10]$ \\ \textbf{Answer:} $\mathfrak c$
\end{itemize}

\question{Determine the cardinality of $\mathcal{P}(\{7,3,10\})$, and compute it. Order the power set by the number of elements, from the least to the greatest. If the numbers of elements for multiple subsets are the same, it doesn't matter what order they're in.}{5}

\textbf{Answer:}\\
$\lvert \mathcal{P}(\{7,3,10\}) \rvert = 2^{\lvert \{7, 3, 10\}\rvert} = 2^3 = 8$\\
$\mathcal{P}(\{7,3,10\}) = \{\varnothing, \{7\}, \{3\}, \{10\}, \{7, 3\}, \{3, 10\}, \{7, 10\}, \{7, 3, 10\}\}$

\bigskip
\bigskip
\bigskip
\bigskip
\bigskip
\bigskip
\bigskip
\bigskip
\bigskip
\bigskip

\question{Compute all partitions for set $\{9,6,7\}$.}{5}


\textbf{Answer:}\\
$\{\{\{9,6,7\}\}, \{\{9\},\{6,7\}\}, \{\{9,6\},\{7\}\}, \{\{9,7\},\{6\}\}, \{\{9\},\{6\},\{7\}\}\}$

\bigskip
\bigskip
\bigskip
\bigskip
\bigskip
\bigskip
\bigskip
\bigskip
\bigskip

\pagebreak

\question{Compute $\{1,5,9\} \times \{x,y\}$.}{5}


\textbf{Answer:}\\
$\{(1, x), (1, y), (5, x), (5, y), (9, x), (9, y)\}$

\bigskip
\bigskip
\bigskip
\bigskip
\bigskip
\bigskip
\bigskip
\bigskip
\bigskip

\question{How many elements are there in $\mathcal{P}(\mathcal{P}(\emptyset)) \times \emptyset$?}{5}


\textbf{Answer:} 0

\bigskip
\bigskip
\bigskip
\bigskip
\bigskip
\bigskip
\bigskip
\bigskip
\bigskip

\pagebreak

\problem{Functions}{50}

\question{Using the function $ f: D \rightarrow [0,10], f(x)=\frac{3x}{2} + 2x^2$ and $[0,10]$ as the co-domain, give a infinite domain $D$ that makes the function surjective.}{10}


\textbf{Answer:}\\
$[0, 1.892]$

\bigskip
\bigskip
\bigskip
\bigskip
\bigskip
\bigskip
\bigskip
\bigskip
\bigskip

\question{Using the function $ f: [-1,1] \rightarrow D, f(x)= x + 2x^2$ and $[-1,1]$ as the domain, give a infinite co-domain $D$ that makes the function surjective.}{10}

\textbf{Answer:}\\
$[-0.125, 3]$

\bigskip
\bigskip
\bigskip
\bigskip
\bigskip
\bigskip
\bigskip
\bigskip
\bigskip

\question{Why is $f(x) = \frac{1}{x}$ not a function from $\mathbb{R}$ to $\mathbb{R}$?}{10}

\textbf{Answer:}\\
It's not defined for $x = 0$ and it never outputs 0, so it's a function from $\mathbb R - \{0\}$ to $\mathbb R - \{0\}$

\bigskip
\bigskip
\bigskip
\bigskip
\bigskip
\bigskip
\bigskip
\bigskip
\bigskip

\pagebreak

\question{Let $A,B,C$ be some arbitrary none empty domain, and $f, g$ are two functions defined as $f: A \rightarrow B, g: B \rightarrow C$.
Define $\circ$ as an operation that combines two functions: $f\circ g (x) = f(g(x))$.

If $g,f$ are one-to-one, is $g\circ f$ also one-to-one? Use the definition of one-to-one functions to prove or disprove this claim.}{10}

\textbf{Answer:}
\begin{flalign}
    & \forall x_1, x_2 \in B, (g(x_1) = g(x_2)) \Rightarrow x_1 = x_2 & \text{$g$ is 1-1} & \\
    & \forall a, b \in A, (g(f(a)) = g(f(b))) \Rightarrow f(a) = f(b) & \text{Substitution (1)} & \\
    & \forall x_1, x_2 \in A, (f(x_1) = f(x_2)) \Rightarrow x_1 = x_2 & \text{$f$ is 1-1} & \\
    & \forall a, b \in A, (g(f(a)) = g(f(b))) \Rightarrow a = b & \text{Modus Ponens (2, 3)} &
\end{flalign}
\setcounter{equation}{0}
By the definition of injectivity, this means that $g \circ f$ is also 1-1.

\pagebreak

\question{Let $A,B,C$ be some arbitrary none empty domain, and $f, g$ are two functions defined as $f: A \rightarrow B, g: B \rightarrow C$.
Define $\circ$ as an operation that combines two functions: $f\circ g (x) = f(g(x))$.

If $g,g\circ f$ are one-to-one, is $f$ also one-to-one? Use the definition of one-to-one functions to prove or disprove this claim.}{10}

\textbf{Answer:}\\
This can be done via a proof by contradiction. Assume that $f$ is not one-to-one.
\begin{flalign}
    & \neg \forall x_1, x_2 \in A, (f(x_1) = f(x_2)) \Rightarrow x_1 = x_2 & \text{$f$ isn't 1-1} & \\
    & \exists x_1, x_2 \in A, \neg ((f(x_1) = f(x_2)) \Rightarrow x_1 = x_2) & \text{De Morgan's (1)} & \\
    & \exists x_1, x_2 \in A, \neg ((f(x_1) \neq f(x_2)) \lor x_1 = x_2) & \text{Definition of implication (2)} & \\
    & \exists x_1, x_2 \in A, f(x_1) = f(x_2) \land x_1 \neq x_2 & \text{De Morgan's (3)} & \\
    & f(x_1) = f(x_2) & \text{Specialization (4)} \\
    & x_1 \neq x_2 & \text{Specialization (4)} \\
    & \forall a, b \in A, (g(f(a)) = g(f(b))) \Rightarrow a = b & \text{$g \circ f$ is 1-1} & \\
    & g(f(x_1)) = g(f(x_2)) & \text{Substitution (5)} \\
    & x_1 = x_2 & \text{Modus Ponens (7, 8)} \\
    & x_1 \neq x_2 \land x_1 = x_2 & \text{Conjunction (6, 9)}
\end{flalign}
\setcounter{equation}{0}
This produces a contradiction. Therefore, the assumption that $f$ is not one-to-one is false.

\pagebreak

\problem{Relations}{25}

For each of these relations defined on $A \times A$ where $A = \{1, 2, 3, 4\}$, decide whether it is reflexive, whether it is symmetric, and whether it is transitive. Please circle the options if you are writing with pen or pencil or \textbf{bold} if you are using latex.\\\\


\question{$R_1 : \{(x,y) \in A \times A| (y|x) \}$}{5}
\textbf{reflexive} symmetric \textbf{transitive}
\bigskip

\question{$R_2 : \{(1, 1), (1, 2), (2, 1), (2, 2), (3, 3), (4, 4) \}$}{5}
\textbf{reflexive} \textbf{symmetric} \textbf{transitive}
\bigskip

\question{$R_3 : \{(x,y) \in A \times A| x+y > 3 \}$}{5}
reflexive \textbf{symmetric} transitive
\bigskip

\question{$R_4 : \{(1, 2), (2, 3), (3, 4)\}$}{5}
reflexive symmetric transitive 
\bigskip

\question{$R_5 : \{(1, 3), (1, 4), (2, 3), (2, 4), (3, 1), (3, 4)\}$}{5}
reflexive symmetric transitive 
\bigskip


\end{document}

%%% Local Variables:
%%% mode: latex
%%% TeX-master: t
%%% End:

