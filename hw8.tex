\documentclass[leqno]{article}

\setlength{\oddsidemargin}{0in}
\setlength{\textwidth}{6in}
\setlength{\topmargin}{-0.1in}
\setlength{\textheight}{8.2in}

%%%%%%%%%%%%%  IMPORT MACRO FILES AS NEEDED %%%%%%%%%%%
\usepackage{amsgen,amsmath,amstext,amsbsy,amsopn,amssymb,amsthm,stackengine}
\usepackage{array, nicefrac, mathtools}
\usepackage{verbatim}
\usepackage{hyperref}
\usepackage{float,relsize,setspace,enumitem,pbox,cleveref,multicol,multirow}
\usepackage{multido}
\usepackage{bbding} % Has a checkmark symbol reachable through \Checkmark
\usepackage{tikz,mdframed}
% \usepackage{circuitikz}
\usepackage{parskip}

% Theorems, definitions, equations, lemmas
\newtheorem{thm}{Theorem}[section]
\newtheorem{prop}[thm]{Proposition}
\newtheorem{lem}[thm]{Lemma}
\newtheorem{cor}[thm]{Corollary}
\newtheorem{defn}{Definition}
\newtheorem{rem}[thm]{Remark}
\numberwithin{equation}{section}
\newtheorem*{defn*}{Definition} % Theorem environments with no numbering
\newtheorem*{prop*}{Proposition}
\newtheorem*{thm*}{Theorem}
\theoremstyle{definition}
\newtheorem*{fact}{Fact}

% For negation and quantifiers in Discrete Math
\newcommand{\shortsim}{\raise.17ex\hbox{$\scriptstyle \sim$}}
\renewcommand{\neg}{\shortsim}
\renewcommand{\nexists}{\neg(\exists}
\newcommand{\nequiv}{\ensuremath{\not\equiv}}

\newcommand{\myline}[1]{\underline{\hspace{#1}}}
\newcommand*\emptycirc[1][1ex]{\tikz\draw (0,0) circle (#1);} 
\newcounter{parts}
\newcounter{problems}[parts]
\newcounter{questions}[problems]
\newcounter{subquestions}[questions]
\newcommand{\hwpart}[1]{
  \stepcounter{parts}
  \noindent\makebox[\textwidth]{\LARGE \bf Part \arabic{parts} - #1}
  \\
}
\newcommand{\problem}[2]{\stepcounter{problems}
  {\Large \bf \noindent Problem \arabic{problems}: #1 \marginpar{[Total #2 pts]} \\[0.3cm]}}
\newcommand{\question}[2]{\stepcounter{questions}
  {\large (\alph{questions}) #1 \marginpar{[#2 pts]} \\[.3cm]}}
\newcommand{\subquestion}[2]{\stepcounter{subquestions}
  {\hspace{10pt}\emph{(\roman{subquestions}) #1 \marginpar{[#2 pts]} }\\[.3cm]}}

% Solution formatting
\newcommand{\solution}[1]{{\color{red}{#1}}}
% Some standard centering and italicization of text.
\newcommand{\frontrowcenter}[1]{\begin{center}{\em \Large  #1  }\end{center}}

% A blank page
\newcommand{\blankpage}{
\clearpage
\vspace*{\fill}
\begin{minipage}{\textwidth}
  \Large \textbf{THIS PAGE INTENTIONALLY LEFT BLANK}\\
\end{minipage}
\vfill % equivalent to \vspace{\fill}
\clearpage
}

\newcommand{\answerspace}[1]{
  \begin{center}
    \textbf{BEGIN YOUR ANSWER BELOW THIS LINE} \\ \hrulefill \vspace{#1} \\ \hrulefill
  \end{center}
}

\newcommand{\answerspacefullpage}{
  \begin{center}
    \textbf{BEGIN YOUR ANSWER BELOW THIS LINE} \\ \hrulefill \pagebreak
  \end{center}
}

\newcommand{\additionalanswerspace}[1]{
  \begin{center}
    \textbf{CONTINUE YOUR ANSWER BELOW THIS LINE } \\ \hrulefill \vspace{#1} \\ \hrulefill
  \end{center}
}

\newcommand{\additionalanswerspacefullpage}{
  \begin{center}
    \textbf{CONTINUE YOUR ANSWER BELOW THIS LINE} \\ \hrulefill \pagebreak
  \end{center}
}

\newcommand{\freespace}[1]{
  \begin{center}
    \large \textbf{SCRAP SPACE BELOW} \\
    \hrulefill
    \pagebreak
  \end{center}
}

% Centered line
\newcommand{\mycenterline}[1]{
  \begin{center}
    \myline{#1}
  \end{center}
}

% Space for T/F:
\newcommand{\tfline}{\myline{.5cm}}

% For quick parenthesized and italicized point annotation.
\newcommand{\pts}[1]{{\em (#1 pts)}}
\newcommand{\onept}{{\em (1 pt)}}

% \item environments coupled with a line at the end, for students to write T and F in.
\newcommand{\tfitem}[1]{\item #1 \null\hfill \framebox(25,25){} \\ \hdashrule{0.95\textwidth}{1pt}{2pt}}
\newcommand{\setitem}[1]{\tfitem{$\curlybraces{#1}$} }
\newcommand{\lineitem}[2]{\item #1 \null \hfill \myline{#2}}

% Some circles and squares for students to fill in.
\newcommand{\whitecircle}[1]{\tikz[baseline=-0.5ex]\draw[black, radius=#1] (0,0) circle ;}
\newcommand{\whitesquare}[1]{\tikz\draw[black] (0,0) rectangl#1, #1) ;}

% Emphasis
\newcommand{\F}{$\mathbf{F}$}
\newcommand{\T}{$\mathbf{T}$}
\newcommand{\False}{\textbf{False}}
\newcommand{\false}{\textbf{false}}
\newcommand{\True}{\textbf{True}}
\newcommand{\true}{\textbf{true}}
\newcommand{\makered}[1]{\textcolor{red}{#1}}
\newcommand{\Rbbst}{\textcolor{red}{Red}-black tree}
\newcommand{\rbbst}{\textcolor{red}{red}-black tree}

\newcommand{\homeworkdata}[4]{
  \begin{mdframed}[linewidth=1pt]
    \noindent\makebox[\textwidth]{\LARGE \bf #1, #2 }
    \\\\
    \noindent\makebox[\textwidth]{\Large \bf  Homework \##3 }
    \\\\
    \noindent\makebox[\textwidth]{\large \bf  Due: #4}
    \\\\
    \noindent\makebox[\textwidth]{\large \bf Homework will not be accepted late}
  \end{mdframed}
  \vspace{40pt}
}

\usepackage{circuitikz}

\setlength{\parindent}{0em}
\setlength{\itemindent}{.5in}

\newcommand{\poneanswer}{%
}
\newcommand{\ptwoanswer}{%
}
\newcommand{\pthreeanswer}{%
}
\newcommand{\pfouranswer}{%
}
\newcommand{\pfiveanswer}{%
}
\newcommand{\psixanswer}{%
}
% \include{solutions}

%%%%%%%%%%%%%%%%%%%%%%%%%%%%%%%%%%%%%%%%%%%%%
%
%  STUDENTS - Your homework begins here.
%
%%%%%%%%%%%%%%%%%%%%%%%%%%%%%%%%%%%%%%%%%%%%%

\counterwithout{equation}{section}

\begin{document}
\pagestyle{empty}

\homeworkdata{CMSC 250}{Fall 2022}{8}{Sunday 13 Nov.\ 11:59pm}

{\Large \bf
  \begin{center}
  IMPORTANT
  \end{center}

  You can write your answers on any paper, either this paper
  or blank paper, or write your answer in Latex (template of this homework can be downloaded through ELMS).
  
  When you upload your document to Gradescope, make sure you tag your questions.

  \begin{center}
    YOU WILL NEED TO TAG YOUR PROBLEMS!!!
  \end{center}

  Problems which are not correctly found will not be graded, this is a zero-tolerance policy. 

  \begin{center}
    IF YOU ARE WORRIED...
  \end{center}

  If you have concerns about tagging your problems,
  We strongly suggest you drop by office hours and do it with a TA present so they can help you through the process,
  just to see how it works. In addition, Gradescope has a tutorial: \url{https://help.gradescope.com/article/ccbpppziu9-student-submit-work#submitting_a_pdf}


}

\pagebreak

\problem{Sum and products}{10}

Calculate the following sum and products:

\question{$\sum_{i = 0}^3 i^2$}{2.5}

$\displaystyle \sum_{i = 0}^3 i^2 = 0 + 1 + 4 + 9 = 14$

\bigskip
\bigskip

\question{$\sum_{i = -2}^{11} \sum_{j = 1}^{4} 2$}{2.5}

$\displaystyle \sum_{i = -2}^{11} \sum_{j = 1}^{4} 2 = \sum_{i = -2}^{11} 2 \cdot 4 = 2 \cdot 4 \cdot (11 + 2 + 1) = 112$

\bigskip
\bigskip

\question{$\sum_{i = 1}^4 (i(\sum_{j = i}^{2} j))$}{2.5}

\newcommand{\paren}[1]{\left(#1\right)}

$\displaystyle \sum_{i = 1}^4 \paren{i\paren{\sum_{j = i}^{2} j}} = 1\paren{\sum_{j = 1}^{2} j} + 2\paren{\sum_{j = 2}^{2} j} + 0 = (1 + 2) + 2(2) = 7$

\bigskip
\bigskip

\question{$\prod_{i = 0}^4 (i-3)$}{2.5}

$\displaystyle \prod_{i = 0}^4 (i-3) = (0 - 3)(1 - 3)(2 - 3)(3 - 3)(4 - 3) = 0$

\pagebreak

\problem{Weak Induction A}{20}
Prove that $3$ divides $n^3 + 2n$ whenever $n$ is a positive integer.
 
This can be proven using weak induction.

\textbf{Base case}: $n = 1$\\
$n^3 + 2n = 1 + 2 = 3$\\
$3 \mid 3$ because $3 = 3(1)$

\textbf{Inductive hypothesis}: Assume that for some arbitrary $k \in \mathbb{N}$ with $k \geq 1$ that $3 \mid k^3 + 2k$.

\textbf{Inductive step}:\\
Goal: $3 \mid (k+1)^3 + 2(k+1)$\\
By the inductive hypothesis, $\exists j \in \mathbb{Z}, k^3 + 2k = 3j$\\
$(k+1)^3 + 2(k+1) = k^3 + 3k^2 + 3k + 2k + 3 = (k^3 + 2k) + 3k^2 + 3k + 3 = 3j + 3(k^2 + k + 1) = 3(j + k^2 + k + 1)$\\
Therefore, $3 \mid 3(j + k^2 + k + 1) = (k+1)^3 + 2(k+1)$.

Thus, we've shown that $\forall k \in \mathbb{N}^{\geq 1}, 3 \mid k^3 + 2k \Rightarrow 3 \mid (k+1)^3 + 2(k+1)$

By the principal of induction, we've shown that 3 divides $n^3+2n$ whenever $n$ is á positive integer.

\pagebreak

\problem{Weak Induction B}{20}

A convex polygon is a polygon where each angles in the polygon is less than $\pi$. Using weak induction, show that for any $n \ge 3$, the sum of angles in a convex polygon with $n$ vertices is $(n-2)\pi$.

This can be proven using weak induction. Define $P(n)$ as the sum of angles in a convex polygon with $n$ vertices being $(n-2)\pi$.

\textbf{Base case:} $n = 3$\\
The sum of the angles in a triangle is $\pi = (3 - 2)\pi$\\
So $P(3)$ is true.

\textbf{Inductive hypothesis:}\\
Assume that for some arbitrary $k \in \mathbb{N}$ with $k \geq 3$ that $P(k)$

\textbf{Inductive step:}\\
We wish to show that $P(k+1)$, i.e. the sum of the angles in a convex polygon with $k + 1$ vertices is $(k+1-2)\pi = (k-1)\pi$.

Suppose there is a convex polygon with $k + 1$ vertices $v_1$, ..., $v_{k+1}$ where $v_i$ is adjacent to $v_{i+1}$ where $1 \leq i < k+1$ and $v_1$ is adjacent to $v_{k+1}$.\\
The polygon can be split into two by drawing a line between $v_1$ and $v_3$, making one triangle with vertices $v_1, v_2, v_3$ and one convex polygon with $k$ vertices $v_1, v_3, v_4, ..., v_{k+1}$.\\
The sum of angles of the original polygon is the sum of angles of these two polygons. The sum of angles of the triangle is $\pi$ and the sum of angles of the $k$-gon is $(k-2)\pi$ by the inductive hypothesis. Adding those together, we get that the sum of the original polygon is $\pi + (k-2)\pi = (k-1)\pi$.

Therefore, we have shown that $P(k+1)$ is true.

\textbf{Conclusion:}\\
By the Principle of Mathematical Induction, the sum of angles in a convex polygon with $n$ vertices is $(n-2)\pi$ for all $n \geq 3$.

\pagebreak

\problem{Strong Induction A}{25}

Let $f_n, n\in \mathbb{N}$ be the sequence satisfying $f_0 = 0, f_1 = 1$ and $f_n = f_{n-1} + f_{n-2}$ for all $n\ge 2$.

Show by induction that:
\begin{itemize}
    \item $f_n^2 + f_{n+1}^2 = f_{2n+1}$ for every $n \in \mathbb{N}$.
    \item $f_{n+2}^2 - f_n^2 = f_{2n+2}$ for every $n \in \mathbb{N}$.
\end{itemize}

This can be proven using strong induction. Define $P(n) \colon (f_n^2 + f_{n+1}^2 = f_{2n+1}) \land (f_{n+2}^2 - f_n^2 = f_{2n+2})$.

\textbf{Base cases:}\\
$n = 0$:\\
$f_0^2 + f_{0+1}^2 = 0^2 + 1^2 = 1 = f_1 = f_{2(0) + 1}$.\\
$f_{0+2}^2 - f{0}^2 = f_{2}^2 - 0 = f_{2(0) + 2}$\\
Thus, $P(0)$ is true

$n = 1$:\\
$f_1^2 + f_{1+1}^2 = 1^2 + 1^2 = 2 = f_1 + f_2 = f_3 = f_{2(1) + 1}$.\\
$f_{1+2}^2 - f_{1}^2 = (f_2 + f_1)^2 - 1 = 3 = f_2 + f_1 + f_2 = f_3 + f_2 = f_4 = f_{2(1) + 2}$\\
Thus, $P(1)$ is true

\textbf{Inductive hypothesis:}\\
Assume for some arbitrary $k \in \mathbb{N}$ with $k \geq 1$ that $\forall i, 0 \leq i \leq k, P(i)$ is true.

\textbf{Inductive step:}\\
We wish to show that $P(k + 1)$, i.e. $f_{k+1}^2 + f_{k+2}^2 = f_{2k + 3}$ and $f_{k+3}^2 - f_{k+1}^2 = f_{2k + 4}$
\begin{flalign*}
    f_{2k+3} &= f_{2k+2} + f_{2k+1} & & \\
    &= f_{k+2}^2 - f_k^2 + f_k^2 + f_{k+1}^2 & \text{(by IH)} \\
    &= f_{k+2}^2 + f_{k+1}^2
\end{flalign*}
\begin{flalign*}
    f_{k+3}^2 - f_{k+1}^2 &= (f_{k+2} + f_{k+1})^2 - f_{k+1}^2 & & \\
    &= f_{k+2}^2 + 2f_{k+2}f_{k+1} + f_{k+1}^2 - f_{k+1}^2 \\
    &= f_{k+2}^2 + 2f_{k+2}f_{k+1} \\
    &= f_{k+2}^2 + 2(f_{k+1} + f_k)f_{k+1} \\
    &= f_{k+2}^2 + 2(f_{k+1}^2 + f_{k}f_{k+1}) \\
    &= f_{k+2}^2 + f_{k+1}^2 + f_{k+1}^2 + 2f_{k}f_{k+1}) \\
    &= f_{2(k+1)+1} + f_{k+1}^2 + 2f_{k}f_{k+1} & \text{(by IH)} \\
    &= f_{2k+3} + f_{k+1}^2 + 2f_{k}f_{k+1} \\
    &= f_{2k+3} + f_{k+1}^2 + 2f_{k}f_{k+1} + f_{k}^2 - f_{k}^2 \\
    &= f_{2k+3} + (f_{k+1} + f_k)^2 - f_{k}^2 \\
    &= f_{2k+3} + f_{k+2}^2 - f_{k}^2 \\
    &= f_{2k+3} + f_{2k+2} & \text{(by IH)} \\
    &= f_{2k+4}
\end{flalign*}
Thus, $P(k+1)$ is true.

\textbf{Conclusion:}\\
By the Principle of Mathematical Induction, we have shown that $P(n)$ holds for every $n \in \mathbb{N}$.

\pagebreak
\bigskip

\problem{Strong Induction B}{25}

Let there be a map with $k$ cities. Any two cities in this map are connected by a road. Using strong induction, show that you can always find a path with no repeating road that travels through all cities.

This can be proven using strong induction. Let $C$ be the set of cities $\{c_1, ..., c_k\}$. Define $P(n)$ to mean that there is a path with no repeating road that travels through cities $c_1, ..., c_n$ and no other cities.

\textbf{Base cases:}\\
$n = 1$: The path is empty, there's only one city.

\textbf{Inductive hypothesis:}\\
Assume that for some arbitrary $j \in \mathbb{N}$ with $j \geq 1$ that $\forall i, 1 \leq i \leq j, P(i)$ is true.

\textbf{Inductive step:}\\
We wish to show that $P(j+1)$ is true.

By the inductive hypothesis, $P(j)$, so there's some path $p$ that travels through cities $c_1, ..., c_n$ and no other cities.\\
Since there's a road $r$ between $c_j$ and $c_{j+1}$, we travel that road at the end of $p$ to obtain a path that travels through $c_1, ..., c_{j+1}$.\\
Adding $r$ does not cause the new path to have repeating roads. If it did, the path would also have to travel through city $c_{j+1}$, but it only travels through $c_1, ..., c_j$. Therefore, $r$ must not already be in path $p$, meaning the new path has no repeating roads.\\
Also, adding $r$ would only cause the path to travel to one new city ($c_{j+1}$), so no other cities outside $c_1, ..., c_{j+1}$ would be visited.

Therefore, $P(j + 1)$ is true.

\textbf{Conclusion:}\\
By the principle of Mathematical Induction, we have shown that there exists a path with no repeating road that travels through $k$ cities where any two cities are connected by a road for any positive integer $k$.

\end{document}

%%% Local Variables:
%%% mode: latex
%%% TeX-master: t
%%% End:

