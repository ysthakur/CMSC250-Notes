\documentclass[12pt, leqno]{article}
\usepackage[utf8]{inputenc}
\usepackage[margin=1in]{geometry}
\usepackage{amssymb}
\usepackage{amsmath}
\usepackage{parskip}

% Line before the therefore in proofs
\newcommand{\proofline}{\rule{0.75in}{0.5pt}}
% For set literals, wraps in {}
\newcommand{\set}[1]{\{#1\}}
% Powerset symbol
\newcommand{\powerset}{\mathcal{P}}
% Cardinality
\newcommand{\card}[1]{\lvert #1 \rvert}
\newcommand{\Z}{\mathbb Z}
\newcommand{\N}{\mathbb N}
\newcommand{\Q}{\mathbb Q}
\newcommand{\R}{\mathbb R}
\newcommand{\evens}{\Z^{\mathrm{even}}}
\newcommand{\odds}{\Z^{\mathrm{odd}}}
\newcommand{\Mod}[1]{\ (\mathrm{mod}\ #1)}

% permutations
\newcommand{\perm}[2]{{}_{#1}\mathrm{P}_{#2}}
% combinations
\newcommand{\comb}[2]{{}_{#1}\mathrm{C}_{#2}}

\title{Quiz 4}
\author{Yash Thakur}
\date{December 5, 2022}

\begin{document}
\counterwithout{equation}{section}

\maketitle

\pagebreak

\section*{Problem 1: Counting}

\begin{enumerate}
    \item[(a)] Strings\\
    Hexadecimal strings with length 3: $16^3$\\
    Hexadecimal strings with length 6: $16^6$\\
    There are $16^3 + 16^6$ hexadecimal strings with either length 3 or 6.
    \item[(b)] Substrings\\
    Each substrings could either have a `C' in it or not\\
    Length of ``CMSC250'': 7\\
    Length of ``MS250'': 5\\
    Number of unique substrings of length 4 without `C': $\perm{5}{4}$\\
    Number of unique substrings of length 4 with one or two `C's: $\comb{6}{3} \cdot 4! \cdot \frac{1}{2}$ (choose 3 letters apart from the `C', permute those 4 letters (including the `C'), then divide by 2 because the two `C's can be swapped)\\
    Add those two together to get the total number:
    \begin{flalign*}
        \perm 5 4 + \comb{6}{3} \cdot 4! \cdot \frac{1}{2} &= \frac{5!}{1!} + \frac{6!}{3!3!} \cdot 4! \cdot \frac{1}{2} &\\
        &= 5! + \frac{6 \cdot 5 \cdot 4}{3!} \cdot 4 \cdot 3 &\\
        &= 5! + 5 \cdot 4 \cdot 4 \cdot 3 &\\
        &= 5! + 2(5 \cdot 4 \cdot 3 \cdot 2) &\\
        &= 5! + 2(5!) &\\
        &= 3 \cdot 5! &\\
        &= 3 \cdot 120 &\\
        &= 360
    \end{flalign*}
    There are 360 unique substrings of length 4 in ``CMSC250''.
    \item[(c)] Groups\\
    Can subtract number of 6 people teams that we don't want from the total number of 6 people teams\\
    Number of 6 people teams that have $A$ and $B$ but not $C$: $\comb{7}{4}$\\
    Total number of 6 people teams: $\comb{10}{6}$\\
    We can make $\comb{10}{6} - \comb{7}{4}$ teams satisfying the criteria.
    \item[(d)] Candy Time\\
    Can use stars and bars: $\comb{12 + 5 - 1}{5 - 1}$\\
    There are $\comb{16}{4}$ ways to buy the candy.
\end{enumerate}

\pagebreak

\section*{Problem 2: Probability}

\begin{enumerate}
    \item[(a)] Strings\\
    Number of anagrams beginning with `C': $\frac{6!}{2}$ (divide by 2 because the `C's can be swapped)\\
    Number of anagrams ending with 0: $\frac{6!}{2}$\\
    Number of anagrams beginning with `C' and ending with 0: $\frac{5!}{2}$\\
    Total number of anagrams: $\frac{7!}{2}$\\
    Probability of an anagram beginning with `C' or ending with 0:
    \begin{flalign*}
        \frac{\frac{6!}{2} + \frac{6!}{2} - \frac{5!}{2}}{\frac{7!}{2}} &= \frac{6! + 6! - 5!}{7!} = \frac{6 + 6 - 1}{7 \cdot 6} = \frac{11}{42}&
    \end{flalign*}
    \item[(b)] Cards\\
    After two draws, number of cards left: 28\\
    Total number of odd cards at start: $3 \cdot 5 = 15$\\
    One of the first two cards must've been odd, so one odd card is gone\\
    Number of odd cards left: $15 - 1 = 14$\\
    Probability of drawing odd card: $\frac{14}{28} = \frac 1 2$
    \item[(c)] Expected value\\
    Expected value from prime number (2, 3, 5): $\frac{3}{6} \cdot 3 = \frac{3}{2}$\\
    Expected value from getting heads after rolling some non-prime number $n$: $\frac{1}{6} \cdot \frac 1 2 \cdot (-n)$\\
    Expected value from getting tails after rolling some non-prime number $n$: $\frac{1}{6} \cdot \frac 1 2 \cdot n$\\
    Total expected value from rolling some non-prime number $n$: $\frac{1}{6} \cdot \frac 1 2 \cdot (-n) + \frac{1}{6} \cdot \frac 1 2 \cdot n = 0$\\
    If you roll a non-prime number, then the expected payoff is 0\\
    So the expected payoff is \$1.5.
    \item[(d)] Bayes\\
    Define $A$ to mean the photo has a platypus and $B$ to mean the AI says the photo has a platypus.\\
    According to question, $P(A) = 0.06$, $P(B \mid A) = 0.92$, and $P(B \mid A^c) = 0.15$\\
    $P(A^c) = 1 - P(A) = 0.94$\\
    $P(B) = P(B \mid A)P(A) + P(B \mid A^c)P(A^c) = (0.92)(0.06) + (0.15)(0.94)$\\
    $P(B^c) = 1 - P(B) = 1 - (0.92)(0.06) - (0.15)(0.94)$\\
    $P(A^c \cap B^c) = P(B^c \mid A^c)P(A^c) = (1 - P(B \mid A^c))P(A^c) = (0.85)(0.94)$\\
    Answer: $\displaystyle P(A^c \mid B^c) = \frac{P(A^c \cap B^c)}{P(B^c)} = \frac{(0.85)(0.94)}{1 - (0.92)(0.06) - (0.15)(0.94)}$
\end{enumerate}

\pagebreak

\section*{Problem 3: Extra Credit}

There are $3^4 = 81$ ways to choose the colors of the cards.\\
The sequence could start at any one of the numbers, so there are 9 possibilities for that\\
So there are $\displaystyle \frac{81 \cdot 9}{4!}$ ways to choose cards satisfying the given conditions (divide by $4!$ because the order doesn't matter).

\end{document}
