\documentclass[12pt]{article}
\usepackage[utf8]{inputenc}
\usepackage[margin=1in]{geometry}
\usepackage{amssymb}
\usepackage{amsmath}
\usepackage{parskip}

% Line before the therefore in proofs
\newcommand{\proofline}{\rule{0.75in}{0.5pt}}
% For set literals, wraps in {}
\newcommand{\set}[1]{\{#1\}}
% Powerset symbol
\newcommand{\powerset}{\mathcal{P}}
% Cardinality
\newcommand{\card}[1]{\lvert #1 \rvert}

\title{More Sets, More Proofs}
\author{Yash Thakur}
\date{September 27, 2022}

\begin{document}

\maketitle

\section*{More Sets}

\subsection*{Cardinality}

The number of items in a set.

\begin{enumerate}
    \item $\card{\set{1, 2, 3, 4, 5}} = 5$
    \item $\card{\varnothing} = 0$
    \item $\card{A \cap B} \leq \card A$
    \item $\card{A \cup B} \leq \card A + \card B$
\end{enumerate}

\subsection*{Partitions}

A set of nonempty, disjoint subsets which when unioned together is equal to the initial set.

The following are \emph{not} partitions of $A = \set{1,2,3,4,5}$:
\begin{enumerate}
    \item $\set{\set{1,2},\set{4,5}}$ (doesn't union together to $A$)
    \item $\set{\set{1,2,3},\set{3,4,5}}$ (not disjoint)
    \item $\set{\set{1,2,3,4,5},\varnothing}$ (has an empty set)
    \item $\set{1,2,3,4,5}$ (not sets, just numbers that happen to be from $A$)
\end{enumerate}

\subsection*{Powersets}

Powerset $\powerset(A) = \set{S \mid S \subseteq A}$ (set of all subsets of $A$)

$\card{\powerset(A)} = 2^{\card A}$

\subsection*{Cartesian Product}

\subsubsection*{Ordered pairs}

Can represent ordered pair $(a, b)$ as $\set{a, \set{a, b}}$

\textbf{Cartesian product} of $A$ and $B$ is set of all ordered pairs between $A$ and $B$\\
$A \times B = \set{(a, b) \mid a \in A \land b \in B}$

$(A \times B) \times C$ is not the same as $A \times B \times C$:\\
$(A \times B) \times C = \set{(u, v) \mid u \in A \times B \land v \in C} = \set{((a1, b1), c1), ((a2, b2), c2), ...}$\\
$A \times B \times C = \set{(x, y, z) \mid x \in A \land y \in B \land z \in C} = \set{(a1, b1, c1), (a2, b2, c2), ...}$

\section*{More Proofs}

\subsubsection*{Existential Instantiation}

If something exists, you can name it. For example, "$x$ is even" can become:\\
$k \in Z$\\
$x = 2k$

\subsection*{Common Mistakes}

\subsubsection*{Arguing from examples}

\subsubsection*{Using the same letter to mean two different things}

\subsubsection*{Jumping to conclusion}

\subsubsection*{Assuming what is to be proved}

\subsubsection*{Confusion between what is known and what is still to be showed}

\subsubsection*{Use of \emph{any} when the correct word is \emph{some}}

\subsubsection*{Misuse of \emph{if}}

\subsection*{Safe Assumptions}

\begin{itemize}
    \item Closure:
    \begin{itemize}
        \item $+, -, *$ on $\mathbf Z$
        \item $+, *$ on $\mathbf N$
        \item $+, *$ on $\mathbf N^{>0}$
        \item $+, -, *$ on $\mathbf R$
        \item $*, /$ on $\mathbf Q^{\neq 0}$
        \item $+, -, *$ on $\mathbf Q$
    \end{itemize}
    \item $\forall x \in \mathbf Z, x \not\in \mathbf Z^{\text{even}} \Rightarrow x \in \mathbf Z^{\text{odd}}$
    \item $\forall x \in \mathbf Z, x \not\in \mathbf Z^{\text{odd}} \Rightarrow x \in \mathbf Z^{\text{even}}$
    \item Unique prime factorization: $\forall n \in \mathbf N^{>1}, \exists!$ set of prime factors when when multiplied together equal $n$
\end{itemize}

\subsection*{Proof Examples}

\end{document}
