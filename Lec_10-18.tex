\documentclass[12pt, leqno]{article}
\usepackage[utf8]{inputenc}
\usepackage[margin=1in]{geometry}
\usepackage{amssymb}
\usepackage{amsmath}
\usepackage{parskip}

% Line before the therefore in proofs
\newcommand{\proofline}{\rule{0.75in}{0.5pt}}
% For set literals, wraps in {}
\newcommand{\set}[1]{\{#1\}}
% Powerset symbol
\newcommand{\powerset}{\mathcal{P}}
% Cardinality
\newcommand{\card}[1]{\lvert #1 \rvert}
\newcommand{\Z}{\mathbb Z}
\newcommand{\N}{\mathbb N}
\newcommand{\Q}{\mathbb Q}
\newcommand{\evens}{\Z^{\mathrm{even}}}
\newcommand{\odds}{\Z^{\mathrm{odd}}}
\newcommand{\Mod}[1]{\ (\mathrm{mod}\ #1)}

\title{Sets and Countability}
\author{Yash Thakur}
\date{October 18, 20, 2022}

\begin{document}
\counterwithout{equation}{section}

\maketitle

Suppose there exist two sets $A$ and $B$. Suppose there also exists a function $f\colon A \mapsto B$.
\begin{itemize}
    \item If $f$ is surjective/onto, we know $|A| \geq |B|$
    \item If $f$ is injective/1-1, we know $|A| \leq |B|$
    \item \emph{Iff} $f$ is bijective, we know $|A| = |B|$
\end{itemize}

Cardinality is also transitive.

Set cardinality classification:
\begin{itemize}
    \item Finite
    \item Countably infinite - a set is countably infinite iff there is a bijective function from it to $\mathbb N$\\
    $\mathbb Z$, $\Z\times\Z$, $\N\times\N$, $\Q$ are all countably infinite.
    \item Uncountably infinite
\end{itemize}

\end{document}
