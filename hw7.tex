

\documentclass[leqno]{article}

\setlength{\oddsidemargin}{0in}
\setlength{\textwidth}{6in}
\setlength{\topmargin}{-0.1in}
\setlength{\textheight}{8.2in}

%%%%%%%%%%%%%  IMPORT MACRO FILES AS NEEDED %%%%%%%%%%%
\usepackage{amsgen,amsmath,amstext,amsbsy,amsopn,amssymb,amsthm,stackengine}
\usepackage{array, nicefrac, mathtools}
\usepackage{verbatim}
\usepackage{hyperref}
\usepackage{float,relsize,setspace,enumitem,pbox,cleveref,multicol,multirow}
\usepackage{multido}
\usepackage{bbding} % Has a checkmark symbol reachable through \Checkmark
\usepackage{tikz,mdframed}
% \usepackage{circuitikz}

% Theorems, definitions, equations, lemmas
\newtheorem{thm}{Theorem}[section]
\newtheorem{prop}[thm]{Proposition}
\newtheorem{lem}[thm]{Lemma}
\newtheorem{cor}[thm]{Corollary}
\newtheorem{defn}{Definition}
\newtheorem{rem}[thm]{Remark}
\numberwithin{equation}{section}
\newtheorem*{defn*}{Definition} % Theorem environments with no numbering
\newtheorem*{prop*}{Proposition}
\newtheorem*{thm*}{Theorem}
\theoremstyle{definition}
\newtheorem*{fact}{Fact}

% For negation and quantifiers in Discrete Math
\newcommand{\shortsim}{\raise.17ex\hbox{$\scriptstyle \sim$}}
\renewcommand{\neg}{\shortsim}
\renewcommand{\nexists}{\neg(\exists}
\newcommand{\nequiv}{\ensuremath{\not\equiv}}

\newcommand{\myline}[1]{\underline{\hspace{#1}}}
\newcommand*\emptycirc[1][1ex]{\tikz\draw (0,0) circle (#1);} 
\newcounter{parts}
\newcounter{problems}[parts]
\newcounter{questions}[problems]
\newcounter{subquestions}[questions]
\newcommand{\hwpart}[1]{
  \stepcounter{parts}
  \noindent\makebox[\textwidth]{\LARGE \bf Part \arabic{parts} - #1}
  \\
}
\newcommand{\problem}[2]{\stepcounter{problems}
  {\Large \bf \noindent Problem \arabic{problems}: #1 \marginpar{[Total #2 pts]} \\[0.3cm]}}
\newcommand{\question}[2]{\stepcounter{questions}
  {\large (\alph{questions}) #1 \marginpar{[#2 pts]} \\[.3cm]}}
\newcommand{\subquestion}[2]{\stepcounter{subquestions}
  {\hspace{10pt}\emph{(\roman{subquestions}) #1 \marginpar{[#2 pts]} }\\[.3cm]}}

% Solution formatting
\newcommand{\solution}[1]{{\color{red}{#1}}}
% Some standard centering and italicization of text.
\newcommand{\frontrowcenter}[1]{\begin{center}{\em \Large  #1  }\end{center}}

% A blank page
\newcommand{\blankpage}{
\clearpage
\vspace*{\fill}
\begin{minipage}{\textwidth}
  \Large \textbf{THIS PAGE INTENTIONALLY LEFT BLANK}\\
\end{minipage}
\vfill % equivalent to \vspace{\fill}
\clearpage
}

\newcommand{\answerspace}[1]{
  \begin{center}
    \textbf{BEGIN YOUR ANSWER BELOW THIS LINE} \\ \hrulefill \vspace{#1} \\ \hrulefill
  \end{center}
}

\newcommand{\answerspacefullpage}{
  \begin{center}
    \textbf{BEGIN YOUR ANSWER BELOW THIS LINE} \\ \hrulefill \pagebreak
  \end{center}
}

\newcommand{\additionalanswerspace}[1]{
  \begin{center}
    \textbf{CONTINUE YOUR ANSWER BELOW THIS LINE } \\ \hrulefill \vspace{#1} \\ \hrulefill
  \end{center}
}

\newcommand{\additionalanswerspacefullpage}{
  \begin{center}
    \textbf{CONTINUE YOUR ANSWER BELOW THIS LINE} \\ \hrulefill \pagebreak
  \end{center}
}

\newcommand{\freespace}[1]{
  \begin{center}
    \large \textbf{SCRAP SPACE BELOW} \\
    \hrulefill
    \pagebreak
  \end{center}
}

% Centered line
\newcommand{\mycenterline}[1]{
  \begin{center}
    \myline{#1}
  \end{center}
}

% Space for T/F:
\newcommand{\tfline}{\myline{.5cm}}

% For quick parenthesized and italicized point annotation.
\newcommand{\pts}[1]{{\em (#1 pts)}}
\newcommand{\onept}{{\em (1 pt)}}

% \item environments coupled with a line at the end, for students to write T and F in.
\newcommand{\tfitem}[1]{\item #1 \null\hfill \framebox(25,25){} \\ \hdashrule{0.95\textwidth}{1pt}{2pt}}
\newcommand{\setitem}[1]{\tfitem{$\curlybraces{#1}$} }
\newcommand{\lineitem}[2]{\item #1 \null \hfill \myline{#2}}

% Some circles and squares for students to fill in.
\newcommand{\whitecircle}[1]{\tikz[baseline=-0.5ex]\draw[black, radius=#1] (0,0) circle ;}
\newcommand{\whitesquare}[1]{\tikz\draw[black] (0,0) rectangl#1, #1) ;}

% Emphasis
\newcommand{\F}{$\mathbf{F}$}
\newcommand{\T}{$\mathbf{T}$}
\newcommand{\False}{\textbf{False}}
\newcommand{\false}{\textbf{false}}
\newcommand{\True}{\textbf{True}}
\newcommand{\true}{\textbf{true}}
\newcommand{\makered}[1]{\textcolor{red}{#1}}
\newcommand{\Rbbst}{\textcolor{red}{Red}-black tree}
\newcommand{\rbbst}{\textcolor{red}{red}-black tree}

\newcommand{\homeworkdata}[4]{
  \begin{mdframed}[linewidth=1pt]
    \noindent\makebox[\textwidth]{\LARGE \bf #1, #2 }
    \\\\
    \noindent\makebox[\textwidth]{\Large \bf  Homework \##3 }
    \\\\
    \noindent\makebox[\textwidth]{\large \bf  Due: #4}
    \\\\
    \noindent\makebox[\textwidth]{\large \bf Homework will not be accepted late}
  \end{mdframed}
  \vspace{40pt}
}

\usepackage{circuitikz}

\setlength{\parindent}{0em}
\setlength{\itemindent}{.5in}

\newcommand{\poneanswer}{%
}
\newcommand{\ptwoanswer}{%
}
\newcommand{\pthreeanswer}{%
}
\newcommand{\pfouranswer}{%
}
\newcommand{\pfiveanswer}{%
}
\newcommand{\psixanswer}{%
}
% \include{solutions}

%%%%%%%%%%%%%%%%%%%%%%%%%%%%%%%%%%%%%%%%%%%%%
%
%  STUDENTS - Your homework begins here.
%
%%%%%%%%%%%%%%%%%%%%%%%%%%%%%%%%%%%%%%%%%%%%%

\begin{document}
\pagestyle{empty}

\homeworkdata{CMSC 250}{Fall 2022}{7}{Sunday 30 Oct.\ 11:59pm}

{\Large \bf
  \begin{center}
  IMPORTANT
  \end{center}

  You can write your answers on any paper, either this paper
  or blank paper, or write your answer in Latex (template of this homework can be downloaded through ELMS).
  
  When you upload your document to Gradescope, make sure you tag your questions.

  \begin{center}
    YOU WILL NEED TO TAG YOUR PROBLEMS!!!
  \end{center}

  Problems which are not correctly found will not be graded, this is a zero-tolerance policy. 

  \begin{center}
    IF YOU ARE WORRIED...
  \end{center}

  If you have concerns about tagging your problems,
  We strongly suggest you drop by office hours and do it with a TA present so they can help you through the process,
  just to see how it works. In addition, Gradescope has a tutorial: \url{https://help.gradescope.com/article/ccbpppziu9-student-submit-work#submitting_a_pdf}


}

\pagebreak

\problem{Countability}{40}

For the following sets, determine whether they are finite, countably infinite or uncountably infinite sets and circle or \textbf{bold} the corresponding option.

\bigskip

\question{$A = \mathbb{R} \times Z$}{8}
finite, countable infinite, \textbf{uncountable infinite}

\bigskip

\question{$B = \mathbb{N}\cap [1, 10]$}{8}
\textbf{finite}, countable infinite, uncountable infinite

\bigskip

\question{$C = \{(x,y) \in N \times N| (y|x)\}$}{8}
finite, \textbf{countable infinite}, uncountable infinite

\bigskip

\question{$D = \{1,2,5,8,9\}$}{8}
\textbf{finite}, countable infinite, uncountable infinite

\bigskip

\question{$E = [5, 6]\cup \mathbb{Z}$}{8}
finite, countable infinite, \textbf{uncountable infinite}

\pagebreak

\problem{Countability proof}{60}

\question{Show that the set $\mathbb{Z}^- \times \mathbb{Z}^+$ is countable, by finding a listing of this set, or by creating a function that maps $\mathbb{Z}^- \times \mathbb{Z}^+$ to $\mathbb{N}$ and prove its bijective.}{20}

\includegraphics{hw7_img/hw7_2a.png}

\pagebreak

\question{Show that the the union of two countable sets is countable, by finding a listing of this set.}{20}

Take two sets $A$ and $B$.

\bigskip

\textbf{Case 1:} $A$ is finite.

In order to list all the elements in $A \cup B$, one can list all the elements of $A$, then list all the elements of $B$.

\bigskip

\textbf{Case 2:} $B$ is finite.

In order to list all the elements in $A \cup B$, one can list all the elements of $B$, then list all the elements of $A$.

\bigskip

\textbf{Case 3:} Both $A$ and $B$ are (countably) infinite.

Since they're countably infinite and a bijection between them and the natural numbers exist, we can write $A$ as $\{a_1, a_2, a_3, ...\}$ and $B$ as $\{b_1, b_2, b_2, ...\}$. $A \cup B$ can be listed by interleaving $A$ and $B$: $a_1, b_1, a_2, b_2, a_3, b_3, ...$

In all cases, a listing of $A \cup B$ can be found, so a union of two countable sets is countable.

\pagebreak

\question{Using Cantor's diagonalization proof, show that \textbf{the set of functions} with domain $\mathbb{Z}$ and codomain $\{1, 4, 7, 9\}$ (i.e. the set of functions that satisfy $f: Z \rightarrow \{1, 4, 7, 9\}$) is uncountable.}{20}
Suppose there is a bijection $f$ between $\mathbb{N}$ and the set of functions with domain $\mathbb{Z}$ and codomain $\{1, 4, 7, 9\}$.

Define a new function $g$ with domain $\mathbb{Z}$ and codomain $\{1, 4, 7, 9\}$ as such:
\begin{itemize}
    \item[] If its input $x$ is negative, $g(x) = 1$
    \item[] Otherwise, let $h$ be $f(x)$. Then
    \begin{itemize}
        \item[] if $h(x) = 1$, $g(x) = 4$
        \item[] if $h(x) = 4$, $g(x) = 7$
        \item[] if $h(x) = 7$, $g(x) = 9$
        \item[] if $h(x) = 9$, $g(x) = 1$
    \end{itemize}
    Thus, $g(x) \neq h(x) = f(x)(x)$
\end{itemize}

Because $g$ has domain $\mathbb{Z}$ and codomain $\{1, 4, 7, 9\}$, there exists a natural number $n$ such that $f(n) = g$. By the definition of $g$, since $n$ is non-negative, $g(n) \neq f(n)(n)$. Since $f(n) = g$, $g(n) \neq g(n)$. This is a contradiction.

Therefore, there cannot exist a bijection between $\mathbb{N}$ and the set of functions with domain $\mathbb{Z}$ and codomain $\{1, 4, 7, 9\}$. Therefore, the set of functions with domain $\mathbb{Z}$ and codomain $\{1, 4, 7, 9\}$ is uncountable.

\end{document}

%%% Local Variables:
%%% mode: latex
%%% TeX-master: t
%%% End:

