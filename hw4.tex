\documentclass[leqno]{article}

\setlength{\oddsidemargin}{0in}
\setlength{\textwidth}{6in}
\setlength{\topmargin}{-0.1in}
\setlength{\textheight}{8.2in}

%%%%%%%%%%%%%  IMPORT MACRO FILES AS NEEDED %%%%%%%%%%%
\usepackage{amsgen,amsmath,amstext,amsbsy,amsopn,amssymb,amsthm,stackengine}
\usepackage{array, nicefrac, mathtools}
\usepackage{verbatim}
\usepackage{hyperref}
\usepackage{float,relsize,setspace,enumitem,pbox,cleveref,multicol,multirow}
\usepackage{multido}
\usepackage{bbding} % Has a checkmark symbol reachable through \Checkmark
\usepackage{tikz,mdframed}
% \usepackage{circuitikz}
\usepackage{changepage}

% Theorems, definitions, equations, lemmas
\newtheorem{thm}{Theorem}[section]
\newtheorem{prop}[thm]{Proposition}
\newtheorem{lem}[thm]{Lemma}
\newtheorem{cor}[thm]{Corollary}
\newtheorem{defn}{Definition}
\newtheorem{rem}[thm]{Remark}
\numberwithin{equation}{section}
\newtheorem*{defn*}{Definition} % Theorem environments with no numbering
\newtheorem*{prop*}{Proposition}
\newtheorem*{thm*}{Theorem}
\theoremstyle{definition}
\newtheorem*{fact}{Fact}

% For negation and quantifiers in Discrete Math
\newcommand{\shortsim}{\raise.17ex\hbox{$\scriptstyle \sim$}}
\renewcommand{\neg}{\shortsim}
\renewcommand{\nexists}{\neg(\exists}
\newcommand{\nequiv}{\ensuremath{\not\equiv}}

\newcommand{\myline}[1]{\underline{\hspace{#1}}}
\newcounter{parts}
\newcounter{problems}[parts]
\newcounter{questions}[problems]
\newcounter{subquestions}[questions]
\newcommand{\hwpart}[1]{
  \stepcounter{parts}
  \noindent\makebox[\textwidth]{\LARGE \bf Part \arabic{parts} - #1}
  \\
}
\newcommand{\problem}[2]{\stepcounter{problems}
  {\Large \bf \noindent Problem \arabic{problems}: #1 \marginpar{[Total #2 pts]} \\[0.3cm]}}
\newcommand{\question}[2]{\stepcounter{questions}
  {\large (\alph{questions}) #1 \marginpar{[#2 pts]} \\[.3cm]}}
\newcommand{\subquestion}[2]{\stepcounter{subquestions}
  {\hspace{10pt}\emph{(\roman{subquestions}) #1 \marginpar{[#2 pts]} }\\[.3cm]}}

% Solution formatting
\newcommand{\solution}[1]{{\color{red}{#1}}}
% Some standard centering and italicization of text.
\newcommand{\frontrowcenter}[1]{\begin{center}{\em \Large  #1  }\end{center}}

% A blank page
\newcommand{\blankpage}{
\clearpage
\vspace*{\fill}
\begin{minipage}{\textwidth}
  \Large \textbf{THIS PAGE INTENTIONALLY LEFT BLANK}\\
\end{minipage}
\vfill % equivalent to \vspace{\fill}
\clearpage
}

\newcommand{\answerspace}[1]{
  \begin{center}
    \textbf{BEGIN YOUR ANSWER BELOW THIS LINE} \\ \hrulefill \vspace{#1} \\ \hrulefill
  \end{center}
}

\newcommand{\answerspacefullpage}{
  \begin{center}
    \textbf{BEGIN YOUR ANSWER BELOW THIS LINE} \\ \hrulefill \pagebreak
  \end{center}
}

\newcommand{\additionalanswerspace}[1]{
  \begin{center}
    \textbf{CONTINUE YOUR ANSWER BELOW THIS LINE } \\ \hrulefill \vspace{#1} \\ \hrulefill
  \end{center}
}

\newcommand{\additionalanswerspacefullpage}{
  \begin{center}
    \textbf{CONTINUE YOUR ANSWER BELOW THIS LINE} \\ \hrulefill \pagebreak
  \end{center}
}

\newcommand{\freespace}[1]{
  \begin{center}
    \large \textbf{SCRAP SPACE BELOW} \\
    \hrulefill
    \pagebreak
  \end{center}
}

% Centered line
\newcommand{\mycenterline}[1]{
  \begin{center}
    \myline{#1}
  \end{center}
}

% Space for T/F:
\newcommand{\tfline}{\myline{.5cm}}

% For quick parenthesized and italicized point annotation.
\newcommand{\pts}[1]{{\em (#1 pts)}}
\newcommand{\onept}{{\em (1 pt)}}

% \item environments coupled with a line at the end, for students to write T and F in.
\newcommand{\tfitem}[1]{\item #1 \null\hfill \framebox(25,25){} \\ \hdashrule{0.95\textwidth}{1pt}{2pt}}
\newcommand{\setitem}[1]{\tfitem{$\curlybraces{#1}$} }
\newcommand{\lineitem}[2]{\item #1 \null \hfill \myline{#2}}

% Some circles and squares for students to fill in.
\newcommand{\whitecircle}[1]{\tikz[baseline=-0.5ex]\draw[black, radius=#1] (0,0) circle ;}
\newcommand{\whitesquare}[1]{\tikz\draw[black] (0,0) rectangl#1, #1) ;}

% Emphasis
\newcommand{\F}{$\mathbf{F}$}
\newcommand{\T}{$\mathbf{T}$}
\newcommand{\False}{\textbf{False}}
\newcommand{\false}{\textbf{false}}
\newcommand{\True}{\textbf{True}}
\newcommand{\true}{\textbf{true}}
\newcommand{\makered}[1]{\textcolor{red}{#1}}
\newcommand{\Rbbst}{\textcolor{red}{Red}-black tree}
\newcommand{\rbbst}{\textcolor{red}{red}-black tree}

\newcommand{\homeworkdata}[4]{
  \begin{mdframed}[linewidth=1pt]
    \noindent\makebox[\textwidth]{\LARGE \bf #1, #2 }
    \\\\
    \noindent\makebox[\textwidth]{\Large \bf  Homework \##3 }
    \\\\
    \noindent\makebox[\textwidth]{\large \bf  Due: #4}
    \\\\
    \noindent\makebox[\textwidth]{\large \bf Homework will not be accepted late}
  \end{mdframed}
  \vspace{40pt}
}

\usepackage{circuitikz}

\setlength{\parindent}{0em}
\setlength{\itemindent}{.5in}

\newcommand{\poneanswer}{%
}
\newcommand{\ptwoanswer}{%
}
\newcommand{\pthreeanswer}{%
}
\newcommand{\pfouranswer}{%
}
\newcommand{\pfiveanswer}{%
}
\newcommand{\psixanswer}{%
}
% \include{solutions}

%%%%%%%%%%%%%%%%%%%%%%%%%%%%%%%%%%%%%%%%%%%%%
%
%  STUDENTS - Your homework begins here.
%
%%%%%%%%%%%%%%%%%%%%%%%%%%%%%%%%%%%%%%%%%%%%%

\begin{document}
\pagestyle{empty}

\homeworkdata{CMSC 250}{Fall 2022}{4}{Sunday 2 Oct.\ 11:59pm}

{\Large \bf
  \begin{center}
  IMPORTANT
  \end{center}

  You can write your answers on any paper, either this paper
  or blank paper, or write your answer in Latex (template of this homework can be downloaded through ELMS).
  
  When you upload your document to Gradescope, make sure you tag your questions.

  \begin{center}
    YOU WILL NEED TO TAG YOUR PROBLEMS!!!
  \end{center}

  Problems which are not correctly found will not be graded, this is a zero-tolerance policy. 

  \begin{center}
    IF YOU ARE WORRIED...
  \end{center}

  If you have concerns about tagging your problems,
  We strongly suggest you drop by office hours and do it with a TA present so they can help you through the process,
  just to see how it works. In addition, Gradescope has a tutorial: \url{https://help.gradescope.com/article/ccbpppziu9-student-submit-work#submitting_a_pdf}


}

\pagebreak


\problem{Modular arithmetics: A}{10}
Decide whether the following statements are true or false.

\bigskip

\question{$4 | 0$?}{2}

\textbf{Answer: True}
\bigskip

\question{$3 | 28$?}{2}

\textbf{Answer: False}
\bigskip

\question{$9 | 63$?}{2}

\textbf{Answer: True}
\bigskip

\question{$2 \equiv 6 (\text{mod } 3)?$}{2}

\textbf{Answer: False}
\bigskip

\question{$-1 \equiv 34 (\text{mod } 5)?$}{2}

\textbf{Answer: True}

\pagebreak


\problem{Modular arithmetics: B}{10}
Answer the following questions. You can show your work / reasoning for partial credit if the final answer is incorrect. 

\bigskip

\question{Explain why $75\equiv 40 \pmod{7}$.}{5}

\textbf{Answer:}\\
$75\equiv 40 \pmod{7} \Leftrightarrow 7 | (75 - 40) \Leftrightarrow 7 | 35 \Leftrightarrow 7 | 5 \cdot 7 \Leftrightarrow 1$
\bigskip
\bigskip
\bigskip

\question{If $m \equiv 2 \pmod{5}$, then what is $m^2 \bmod 5$?}{5}

\textbf{Answer:}\\
$m = 5k + 2$\\
$m^2 \equiv (5k+2)^2 \equiv 25k^2 + 20k + 4 \equiv 4 \pmod{5}$\\
$m^2 \bmod 5 = 4$
\bigskip
\bigskip
\bigskip

\pagebreak

\problem{negate quantifiers}{10}
Write formal negations (simplify the negation as much as possible, until the negation is only in front of variables) of the following quantified statement. You can show your work to get partial credits if the final result is incorrect.

\bigskip

\question{$\forall x \in \mathbb{R} (x > 3 \Rightarrow x^2 > 9)$}{5}

\textbf{Answer: }\\
\begin{align*}
	\neg \forall x \in \mathbb{R} (x > 3 \Rightarrow x^2 > 9) &\equiv \exists x \in \mathbb{R} \neg (x > 3 \Rightarrow x^2 > 9) & (\text{Negation of universal}) \\
	&\equiv \exists x \in \mathbb{R} \neg (\neg (x > 3) \lor (x^2 > 9)) & (\text{Definition of implication}) \\
	&\equiv \exists x \in \mathbb{R} ((x > 3) \land \neg (x^2 > 9)) & (\text{De Morgan's}) \\
	&\equiv \exists x \in \mathbb{R} ((x > 3) \land (x^2 \leq 9)) & (\text{Because})
\end{align*}
\bigskip
\bigskip
\bigskip
\bigskip
\bigskip
\bigskip
\bigskip
\bigskip
\bigskip
\bigskip
\bigskip
\bigskip
\bigskip

\question{$\exists x,y,z \in \mathbb{Z} ((2|(a-b) \land 2|(b-c)) \Leftrightarrow 2|(a-c))$}{5}

\textbf{Answer:}
\begin{align*}
	&\neg \exists x,y,z \in \mathbb{Z} ((2|(a-b) \land 2|(b-c)) \Leftrightarrow 2|(a-c)) \\
	&\equiv \forall x,y,z \in \mathbb{Z} \,\neg ((2|(a-b) \land 2|(b-c)) \Leftrightarrow 2|(a-c)) & (\text{$\exists$ negation}) \\
	&\equiv \forall x,y,z \in \mathbb{Z} \,\neg (((2|(a-b) \land 2|(b-c)) \Rightarrow 2|(a-c)) \land (2|(a-c) \Rightarrow (2|(a-b) \land 2|(b-c)))) & (\text{Definition of $\Leftrightarrow$}) \\
	&\equiv \forall x,y,z \in \mathbb{Z} \,(\neg ((2|(a-b) \land 2|(b-c)) \Rightarrow 2|(a-c)) \lor \neg (2|(a-c) \Rightarrow (2|(a-b) \land 2|(b-c)))) & (\text{De Morgan's}) \\
	&\equiv \forall x,y,z \in \mathbb{Z} \,(\neg (\neg (2|(a-b) \land 2|(b-c)) \lor 2|(a-c)) \lor \neg (\neg(2|(a-c)) \lor (2|(a-b) \land 2|(b-c)))) & (\text{Definition of $\Rightarrow$}) \\
	&\equiv \forall x,y,z \in \mathbb{Z} \,(((2|(a-b) \land 2|(b-c)) \land \neg(2|(a-c))) \lor ((2|(a-c)) \land \neg(2|(a-b) \land 2|(b-c)))) & (\text{De Morgan's}) \\
	&\equiv \forall x,y,z \in \mathbb{Z} \,(((2|(a-b) \land 2|(b-c)) \land \neg(2|(a-c))) \lor ((2|(a-c)) \land (\neg(2|(a-b)) \lor \neg(2|(b-c))))) & (\text{De Morgan's}) \\
	&\equiv \forall x,y,z \in \mathbb{Z} \,(((2|(a-b) \land 2|(b-c)) \land (2\nmid(a-c))) \lor ((2|(a-c)) \land (2\nmid(a-b) \lor 2\nmid(b-c)))) & (\text{$\neg(p\mid q)$ $\rightarrow$ $p \nmid q$})
\end{align*}
\bigskip
\bigskip
\bigskip
\bigskip

\pagebreak

\problem{Sets: subsets}{5}
Is either A or B a subset, or a proper subset of the other?

In set operaitons, we treat $\backslash$ as set minus operation. In other words, given two sets $A, B$, $A\backslash B = A-B$.

\bigskip

\question{$A = \{1,2,3,4,5\}\cup\{3,7,9,10\}, B = \{1,2,3,4,5\}\cap\{3,7,9,10\}$}{1}

\textbf{Answer:} $A \subseteq B \land B \subseteq A$

\bigskip

\question{$A = Z \backslash [3, 40], B = \mathbb{R} \cap \mathbb{Z}$}{1}

\textbf{Answer:} $A \subset B$

\bigskip

\question{$A = \{ \forall x \in \mathbb{Z} | x \equiv 2 (\text{mod } 8)\}, B = \{\forall x \in \mathbb{R}|x \equiv 2 (\text{mod } 4)\}$}{1}

\textbf{Answer:} $A \subset B$

\bigskip

\question{$A = \{16, \{4\}\}, B = \emptyset$}{1}

\textbf{Answer:} $B \subset A$

\bigskip

\question{$A = \mathbb{R} \backslash \mathbb{Z}, B = \mathbb{R} \backslash \mathbb{Q}$}{1}

\textbf{Answer:} $B \subset A$

\pagebreak

\problem{Sets: Elements}{5}
Is A an element of the set B?
\bigskip

\question{$A = \emptyset, B = \mathbb{R}$}{1}

\textbf{Answer: No}

\bigskip
\question{$A = 3.4, B = \{x \in \mathbb{Z} | -2 < x < 5\}$}{1}

\textbf{Answer: No}

\bigskip

\question{$A = 10.2, B = [1,20]$}{1}

\textbf{Answer: Yes}

\bigskip

\question{$A = \{2\}, B = \{2,3,4\}$}{1}

\textbf{Answer: No}

\bigskip

\question{$A = 2, B = \{2,3,4\}$}{1}

\textbf{Answer: Yes}

\bigskip


\pagebreak

\problem{Direct Proof: A}{10}%
Using direct proof, show that:

For all integers $a, b$, and $c$, if $a | b$ and $a | c$ then $a |(b - c)$.

\textbf{Answer:}
\counterwithout{equation}{section}
\begin{flalign}
    & a, b, c \in \mathbb{Z} & \text{Assumption} & \\
    & a \mid b & \text{Assumption} & \\
    & a \mid c & \text{Assumption} & \\
    & \exists k \in \mathbb{Z}, b = ka & \text{Definition of $\mid$ (2)} & \\
    & \exists l \in \mathbb{Z}, c = la & \text{Definition of $\mid$ (3)} & \\
    & a \mid (k-l)a & \text{Definition of $\mid$} \\
    & a \mid ka - la & \text{Distributive (6)} \\
    & a \mid (b - c) & \text{Substitution (4, 5, 7)}
\end{flalign}
\setcounter{equation}{0}

\pagebreak


\problem{Direct Proof: B}{10}%
Using direct proof, show that:

The product of any two rational numbers is a rational number.

\textbf{Answer:}
\begin{flalign}
    & p, q \in \mathbb{Q} & \text{Assumption} & \\
    & \exists a, b \in \mathbb{Z}, p = \frac a b & \text{Definition of $\mathbb{Q}$ (1)} \\
    & \exists c, d \in \mathbb{Z}, q = \frac c d & \text{Definition of $\mathbb{Q}$ (1)} \\
    & ac \in \mathbb{Z} & \text{Multiplication closed over $\mathbb{Z}$} \\
    & bd \in \mathbb{Z} & \text{Multiplication closed over $\mathbb{Z}$} \\
    & pq = \frac{ac}{bd} & \text{Substitution (2, 3)} \\
    & pq \in \mathbb Q & \text{Definition of $\mathbb Q$ (4, 5, 6)}
\end{flalign}
\setcounter{equation}{0}

\pagebreak

\problem{Indirect Proof: Contradiction A}{10}%
Using contradiction, show that:

There is no greatest odd integer.

\textbf{Answer:}
Assume there is a greatest odd integer.
\begin{flalign}
    & n \in O \text{ where } O = \{\forall x \in \mathbb Z \mid \exists k \in \mathbb Z (x = 2k + 1)\} & \text{Assumption} & \\
    & \forall x \in O, n \geq x & \text{Assumption} \\
    & \exists k \in \mathbb Z, n = 2k + 1 & \text{Definition of $O$ (1)} \\
    & \text{Let } m = n + 2 & \text{Because I can} \\
    & m = 2k + 1 + 2 & \text{Substitution (3)} \\
    & m = 2(k + 1) + 1 & \text{Arithmetic (5)} \\
    & m \in O & \text{Definition of $O$ (1, 6)} \\
    & n \geq m & \text{Universal elimination (2)} \\
    & n < m & \text{Definition of $m$ (4)} \\
    & (n < m) \land (n \geq m) & \text{Conjunction (8, 9)} \\
    & \bot & \text{Contradiction (10)}
\end{flalign}
\setcounter{equation}{0}
Therefore, it cannot be true that there is a greatest odd integer.

\pagebreak


\problem{Indirect Proof: Contradiction B}{10}%
Using contradiction, show that:

Every integer greater than 11 is a sum of two composite numbers.

\textbf{Answer:}\\
This can be done via a proof by contradiction. First, we assume the negation is true, that is, assume $\neg \forall n \in \mathbb Z^{>11}, \exists x, y \in \mathbb N, x \not\in \mathbb P \land y \not\in \mathbb P \land (x + y = n)$.
\begin{flalign*}
    & \neg \forall n \in \mathbb Z^{>11}, \exists x, y \in \mathbb N, x \not\in \mathbb P \land y \not\in \mathbb P \land (x + y = n) & \text{Assumption} & \\
    & \equiv \exists n \in \mathbb Z^{>11}, \neg \exists x, y \in \mathbb N, x \not\in \mathbb P \land y \not\in \mathbb P \land (x + y = n) & \text{Universal negation} & \\
    & \equiv \exists n \in \mathbb Z^{>11}, \forall x, y \in \mathbb N, \neg(x \not\in \mathbb P \land y \not\in \mathbb P \land (x + y = n)) & \text{Existential negation} & \\
    & \equiv \exists n \in \mathbb Z^{>11}, \forall x, y \in \mathbb N, x \in \mathbb P \lor y \in \mathbb P \lor (x + y \neq n) & \text{De Morgan's x2}
\end{flalign*}
\setcounter{equation}{0}
Let us take this integer $n$ that is greater than 11 and isn't a sum of two composite numbers.
\bigskip
\begin{adjustwidth}{0.75cm}{}
\textbf{Case 1:} $n$ is even
\begin{flalign}
    & \exists k \in \mathbb Z, n = 2k & \text{Definition of even} & \\
    & 2k > 11 & \text{Substitution (1)} \\
    & k > \frac{11}{2} & \text{Arithmetic (2)} \\
    & n = 2k - 6 + 6 & \text{Arithmetic (1)} \\
    & n = 2(k - 3) + 6 & \text{Factoring (4)} \\
    & k - 3 > \frac{5}{2} & \text{Arithmetic (3)} \\
    & k - 3 > 1 & \text{Transitive property (6)} \\
    & 2(k - 3) \not\in \mathbb P & \text{2 factors $>1$ other than itself} \\
    & 6 \not\in \mathbb P & \text{Has factors 2 and 3}
\end{flalign}
\setcounter{equation}{0}
Therefore, $n$ is a sum of two composite numbers, $2(k - 3)$ and 6.
\end{adjustwidth}
\bigskip
\begin{adjustwidth}{0.75cm}{}
\textbf{Case 2:} $n$ is odd
\begin{flalign}
    & \exists k \in \mathbb Z, n = 2k + 1 & \text{Definition of odd} & \\
    & 2k + 1 > 11 & \text{Substitution (1)} \\
    & k > 5 & \text{Arithmetic (2)} \\
    & n = 2k - 8 + 8 + 1 & \text{Arithmetic (1)} \\
    & n = 2(k - 4) + 9 & \text{Arithmetic (4)} \\
    & k - 4 > 1 & \text{Arithmetic (3)} \\
    & 2(k - 4) \not\in \mathbb P & \text{2 factors $>1$ other than itself} \\
    & 9 \not\in \mathbb P & \text{Has factors 3 and 3}
\end{flalign}
\setcounter{equation}{0}
Therefore, $n$ is a sum of two composite numbers, $2(k - 4)$ and 9.
\end{adjustwidth}
\bigskip
In both cases, $n$ is a sum of two composite numbers. However, it follows from the assumption that $n$ isn't a sum of two composite numbers. This is a contradiction. Therefore, the negation of the assumption must be true: all integers greater than 11 are sums of two composite numbers.

\pagebreak

\problem{Indirect Proof: Contraposition A}{10}%
Using Contraposition, show that:

 For all integers $m$ and $n$, if $mn$ is even then $m$ is even or $n$
is even.

\textbf{Answer:}\\
Goal: $\forall m, n \in \mathbb Z, \neg(m \in \mathbb Z^{even} \lor n \in \mathbb Z^{even}) \Rightarrow \neg(mn \in \mathbb Z^{even})$
\begin{flalign}
    & \text{Let } m, n \in \mathbb Z & \text{Assumption} & \\
    & \neg(m \in \mathbb Z^{\text{even}} \lor n \in \mathbb Z^{\text{even}}) & \text{Assumption} \\
    & m \not\in \mathbb Z^{\text{even}} \land n \not\in \mathbb Z^{\text{even}} & \text{De Morgan's} \\
    & m \not\in \mathbb Z^{\text{even}} & \text{Conjunction elimination} \\
    & n \not\in \mathbb Z^{\text{even}} & \text{Conjunction elimination} \\
    & m \in \mathbb Z^{\text{odd}} & \text{Parity} \\
    & n \in \mathbb Z^{\text{odd}} & \text{Parity} \\
    & \exists k \in \mathbb Z, m = 2k + 1 & \text{Definition of odd} \\
    & \exists j \in \mathbb Z, n = 2j + 1 & \text{Definition of odd} \\
    & mn = (2k + 1)(2j + 1) & \text{Substitution} \\
    & mn = 4kj + 2k + 2j + 1 & \text{Simplification} \\
    & mn = 2(kj + k + j) + 1 & \text{Factoring} \\
    & mn \in \mathbb Z^{\text{odd}} & \text{Definition of odd} \\
    & mn \not\in \mathbb Z^{\text{even}} & \text{Parity}
\end{flalign}
\setcounter{equation}{0}

Since $\forall m, n \in \mathbb Z, \neg(m \in \mathbb Z^{even} \lor n \in \mathbb Z^{even}) \Rightarrow \neg(mn \in \mathbb Z^{even})$, the contrapositive is also true: $\forall m, n \in \mathbb Z, mn \in \mathbb Z^{\text{even}} \Rightarrow (m \in \mathbb Z^{\text{even}} \lor n \in \mathbb Z^{\text{even}})$


\pagebreak

\problem{Indirect Proof: Contraposition B}{10l}%
Using Contraposition, show that:

For all real numbers $r$, if $r^2$ is irrational then $r$ is irrational.

\textbf{Answer:}\\
Goal: $\forall r \in \mathbb R, r \in \mathbb R \Rightarrow r^2 \in \mathbb R$
\begin{flalign}
    & r \in \mathbb R & \text{Assumption} & \\
    & \exists p,q \in \mathbb Z, r = \frac p q & \text{Definition of }\mathbb R \\
    & r^2 = \left(\frac p q\right)^2 & \text{Substitution (2)} \\
    & r^2 = \frac{p \cdot p}{q \cdot q} & \text{Arithmetic (3)} \\
    & p \cdot p \in \mathbb Z & \mathbb Z \text{ closed under multiplication} \\
    & q \cdot q \in \mathbb Z & \mathbb Z \text{ closed under multiplication} \\
    & r^2 \in \mathbb R & \text{Definition of $\mathbb R$ (4, 5, 6)}
\end{flalign}
\setcounter{equation}{0}
Since we've proven that for any real number $r$, if $r$ is rational, $r^2$ is also rational, the contrapositive is also true: if $r^2$ is irrational, then $r$ is irrational.

\pagebreak

\pagebreak



\end{document}

%%% Local Variables:
%%% mode: latex
%%% TeX-master: t
%%% End:

