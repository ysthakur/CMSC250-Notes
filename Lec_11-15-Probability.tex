\documentclass[12pt, leqno]{article}
\usepackage[utf8]{inputenc}
\usepackage[margin=1in]{geometry}
\usepackage{amssymb}
\usepackage{amsmath}
\usepackage{parskip}

% Line before the therefore in proofs
\newcommand{\proofline}{\rule{0.75in}{0.5pt}}
% For set literals, wraps in {}
\newcommand{\set}[1]{\{#1\}}
% Powerset symbol
\newcommand{\powerset}{\mathcal{P}}
% Cardinality
\newcommand{\card}[1]{\lvert #1 \rvert}
\newcommand{\Z}{\mathbb Z}
\newcommand{\N}{\mathbb N}
\newcommand{\Q}{\mathbb Q}
\newcommand{\R}{\mathbb R}
\newcommand{\evens}{\Z^{\mathrm{even}}}
\newcommand{\odds}{\Z^{\mathrm{odd}}}
\newcommand{\Mod}[1]{\ (\mathrm{mod}\ #1)}

% permutations
\newcommand{\perm}[2]{{}_{#1}\mathrm{P}_{#2}}
% combinations
\newcommand{\comb}[2]{{}_{#1}\mathrm{C}_{#2}}

\title{Probability}
\author{Yash Thakur}
\date{November 15, 2022}

\begin{document}
\counterwithout{equation}{section}

\maketitle

\textbf{Definition:} If $S$ is a finite sample space in which all outcomes are equally likely and $E$ is an event in $S$, then the \textbf{probability} of $E$ is:
\[P(E) = \frac{\text{the number of outcomes in }E}{\text{the total number of outcomes in }S}\]

The probability will always be between 0 and 1 ($0 \leq P(E) \leq 1$).

\textbf{Probability function} maps all events in a sample space $S$ to $\R$

$P(A^c) = 1 - P(A)$

If $A$ and $B$ are disjoint: $P(A \cup B) = P(A) + P(B)$\\
In general: $P(A \cup B) = P(A) + P(B) - P(A \cap B)$

Can negate a problem to make it easier

\subsection*{Expected value}

The \textbf{expected value} of an event is the probability of an event times its value:
\[E = \sum_{i=1}^n p_i a_i\]

\end{document}
