\documentclass[12pt, leqno]{article}
\usepackage[utf8]{inputenc}
\usepackage[margin=1in]{geometry}
\usepackage{amssymb}
\usepackage{amsmath}
\usepackage{parskip}

% Line before the therefore in proofs
\newcommand{\proofline}{\rule{0.75in}{0.5pt}}
% For set literals, wraps in {}
\newcommand{\set}[1]{\{#1\}}
% Powerset symbol
\newcommand{\powerset}{\mathcal{P}}
% Cardinality
\newcommand{\card}[1]{\lvert #1 \rvert}
\newcommand{\Z}{\mathbb Z}
\newcommand{\Q}{\mathbb Q}
\newcommand{\evens}{\Z^{\mathrm{even}}}
\newcommand{\odds}{\Z^{\mathrm{odd}}}
\newcommand{\Mod}[1]{\ (\mathrm{mod}\ #1)}

\title{Quiz 2}
\author{Yash Thakur}
\date{October 10, 2022}

\begin{document}
\counterwithout{equation}{section}

\section*{Problem 1: Proofs}

Prove using a direct proof:

\begin{center}
    $\forall a, b, c, d, m \in \Z, (m \geq 2 \land a \equiv b \pmod{m} \land c \equiv d \pmod{m}) \Rightarrow (a + c \equiv d + b \pmod{m})$
\end{center}

\begin{flalign}
    & m \geq 2 \land a \equiv b \Mod{m} \land c \equiv d \Mod{m} & \text{Assumption} & \\
    & m \geq 2 & \text{Specialization (1)} & \\
    & a \equiv b \Mod{m} & \text{Specialization (1)} & \\
    & c \equiv d \Mod{m} & \text{Specialization (1)} & \\
    & m \mid (a - b) & \text{Def. of modular equivalence (3)} \\
    & m \mid (c - d) & \text{Def. of modular equivalence (4)} \\
    & \exists k \in \Z, a - b = mk & \text{Definition of $\mid$ (5)} \\
    & \exists j \in \Z, c - d = mj & \text{Definition of $\mid$ (6)} \\
    & a = mk + b & \text{Arithmetic (7)} \\
    & c = mj + d & \text{Arithmetic (8)} \\
    & a + c = mk + b + mj + d & \text{Substitution (9, 10)} \\
    & (a + c) - (d + b) = m(k + j) & \text{Arithmetic (11)} \\
    & k + j \in \Z & \text{$\Z$ closed under addition} \\
    & m \mid (a + c) - (d + b) & \text{Definition of $\mid$ (12, 13)} \\
    & a + c \equiv d + b \Mod{m} & \text{Def. of modular equivalence (14)}
\end{flalign}
\setcounter{equation}{0}

\pagebreak

\section*{Problem 2: Proofs}

Prove using either a proof by contradiction or a proof by contrapositive:

\begin{center}
    $\forall x \in \Z, x \equiv 5 \pmod{2} \Rightarrow x + 1 \in \evens$
\end{center}

This can be done using a proof by contrapositive. The contrapositive of the statement above is $\forall x \in \Z, x + 1 \in \odds \Rightarrow x \not\equiv 5 \pmod{2}$.
\begin{flalign}
    & x + 1 \in \odds & \text{Assumption} & \\
    & \exists k \in \Z, x + 1 = 2k + 1 & \text{Definition of even (1)} & \\
    & (x + 1) - 5 = 2k + 1 - 5 & \text{Arithmetic (2)} \\
    & x - 5 = 2k - 5 & \text{Arithmetic (3)} \\
    & x - 5 = 2(k - 3) + 1 & \text{Factoring (4)} \\
    & k - 3 \in \Z & \text{$\Z$ closed under addition} \\
    & x - 5 \in \odds & \text{Definition of odd (5, 6)} \\
    & x - 5 \not\in \evens & \text{Parity (7)} \\
    & 2 \nmid x - 5 & \text{Definition of $\evens$ (8)} \\
    & x \not\equiv 5 \Mod{2} & \text{Definition of modular equivalence (9)}
\end{flalign}
\setcounter{equation}{0}
Since the contrapositive is true, the original statement is also true.

\pagebreak

\section*{Problem 3: Proofs}

Prove using any of the 3 proof types we have talked about

\begin{center}
    $\forall x \in \Z, 4 \mid (x^2 - 6x)(x^2 + 3)$
\end{center}

This can be done with a direct proof using two cases based on $x$'s parity.

\textbf{Case 1:} $x \in \odds$
\begin{flalign}
    & \exists k \in \Z, x = 2k + 1 & x \in \odds & \\
    & x^2 + 3 = (2k + 1)^2 + 3 & \text{Substitution (1)} \\
    & x^2 + 3 = 4k^2 + 4k + 4 & \text{Arithmetic (2)} \\
    & x^2 + 3 = 4(k^2 + k + 1) & \text{Factoring (3)} \\
    & (x^2 - 6x)(x^2 + 3) = 4(x^2 - 6x)(k^2 + k + 1) & \text{Substitution (4)} & \\
    & (x^2 - 6x)(k^2 + k + 1) \in \Z & \text{$\Z$ closed under $+$ and $\cdot$} & \\
    & 4 \mid (x^2 - 6x)(x^2 + 3) & \text{Definition of $\mid$ (5, 6)}
\end{flalign}
\setcounter{equation}{0}

\textbf{Case 2}: $x \in \evens$
\begin{flalign}
    & \exists k \in \Z, x = 2k & x \in \evens & \\
    & x^2 - 6x = (2k)^2 - 6(2k) & \text{Substitution (1)} \\
    & x^2 - 6x = 4k^2 - 12k & \text{Arithmetic (2)} \\
    & x^2 - 6x = 4(k^2 - 3k) & \text{Factoring (3)} \\
    & (x^2 - 6x)(x^2 + 3) = 4(k^2 - 3k)(x^2 + 3) & \text{Substitution (4)} & \\
    & (k^2 - 3k)(x^2 + 3) \in \Z & \text{$\Z$ closed under $+$ and $\cdot$} & \\
    & 4 \mid (x^2 - 6x)(x^2 + 3) & \text{Definition of $\mid$ (5, 6)}
\end{flalign}
\setcounter{equation}{0}

In both cases, one can conclude that $4 \mid (x^2 - 6x)(x^2 + 3)$. Therefore, the statement holds for all integers.

\pagebreak

\section*{Problem 4: Proofs}

Prove using any of the 3 proof types we have talked about

\begin{center}
    Assuming that $\sqrt 6$ is irrational, prove that $\sqrt 2 + \sqrt 3$ is also irrational.
\end{center}

This can be done using a proof by contradiction. Assume that $\sqrt 2 + \sqrt 3 \in \Q$.
\begin{flalign}
    & \sqrt 6 \not\in \Q & \text{Assumption} & \\
    & \sqrt 2 + \sqrt 3 \in \Q & \text{Assumption} & \\
    & \sqrt 2, \sqrt 3 \in \Q & \text{$\Q$ closed under addition (2)} \\
    & \exists p, q \in \Z, \frac p q = \sqrt 2 & \text{Definition of $\Q$} \\
    & \exists r, s \in \Z, \frac r s = \sqrt 3 & \text{Definition of $\Q$} \\
    & \sqrt 6 = \sqrt{2 \cdot 3} & \text{Arithmetic (1)} \\
    & \sqrt 6 = \sqrt 2 \sqrt 3 & \text{Arithmetic (7)} \\
    & \sqrt 6 = \frac p q \cdot \frac r s & \text{Substitution (8, 4, 5)} \\
    & pr \in \Z & \text{$\Z$ closed under multiplication} \\
    & qs \in \Z & \text{$\Z$ closed under multiplication} \\
    & \sqrt 6 \in \Q & \text{Definition of $\Q$ (8, 9, 10)} \\
    & \sqrt 6 \in \Q \land \sqrt 6 \not\in \Q & \text{Conjunction (1, 11)}
\end{flalign}
This produces a contradiction. Therefore, it is not true that $\sqrt 2 + \sqrt 3 \in \Q$, i.e. $\sqrt 2 + \sqrt 3$ is irrational.

\end{document}
