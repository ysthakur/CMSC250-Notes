\documentclass[leqno]{article}

\setlength{\oddsidemargin}{0in}
\setlength{\textwidth}{6in}
\setlength{\topmargin}{-0.1in}
\setlength{\textheight}{8.2in}

%%%%%%%%%%%%%  IMPORT MACRO FILES AS NEEDED %%%%%%%%%%%
\usepackage{amsgen,amsmath,amstext,amsbsy,amsopn,amssymb,amsthm,stackengine}
\usepackage{array, nicefrac, mathtools}
\usepackage{verbatim}
\usepackage{hyperref}
\usepackage{float,relsize,setspace,enumitem,pbox,cleveref,multicol,multirow}
\usepackage{multido}
\usepackage{bbding} % Has a checkmark symbol reachable through \Checkmark
\usepackage{tikz,mdframed}
% \usepackage{circuitikz}

% Theorems, definitions, equations, lemmas
\newtheorem{thm}{Theorem}[section]
\newtheorem{prop}[thm]{Proposition}
\newtheorem{lem}[thm]{Lemma}
\newtheorem{cor}[thm]{Corollary}
\newtheorem{defn}{Definition}
\newtheorem{rem}[thm]{Remark}
\numberwithin{equation}{section}
\newtheorem*{defn*}{Definition} % Theorem environments with no numbering
\newtheorem*{prop*}{Proposition}
\newtheorem*{thm*}{Theorem}
\theoremstyle{definition}
\newtheorem*{fact}{Fact}

% For negation and quantifiers in Discrete Math
\newcommand{\shortsim}{\raise.17ex\hbox{$\scriptstyle \sim$}}
\renewcommand{\neg}{\shortsim}
\renewcommand{\nexists}{\neg(\exists}
\newcommand{\nequiv}{\ensuremath{\not\equiv}}

\newcommand{\myline}[1]{\underline{\hspace{#1}}}
\newcommand*\emptycirc[1][1ex]{\tikz\draw (0,0) circle (#1);} 
\newcounter{parts}
\newcounter{problems}[parts]
\newcounter{questions}[problems]
\newcounter{subquestions}[questions]
\newcommand{\hwpart}[1]{
  \stepcounter{parts}
  \noindent\makebox[\textwidth]{\LARGE \bf Part \arabic{parts} - #1}
  \\
}
\newcommand{\problem}[2]{\stepcounter{problems}
  {\Large \bf \noindent Problem \arabic{problems}: #1 \marginpar{[Total #2 pts]} \\[0.3cm]}}
\newcommand{\question}[2]{\stepcounter{questions}
  {\large (\alph{questions}) #1 \marginpar{[#2 pts]} \\[.3cm]}}
\newcommand{\subquestion}[2]{\stepcounter{subquestions}
  {\hspace{10pt}\emph{(\roman{subquestions}) #1 \marginpar{[#2 pts]} }\\[.3cm]}}

% Solution formatting
\newcommand{\solution}[1]{{\color{red}{#1}}}
% Some standard centering and italicization of text.
\newcommand{\frontrowcenter}[1]{\begin{center}{\em \Large  #1  }\end{center}}

% A blank page
\newcommand{\blankpage}{
\clearpage
\vspace*{\fill}
\begin{minipage}{\textwidth}
  \Large \textbf{THIS PAGE INTENTIONALLY LEFT BLANK}\\
\end{minipage}
\vfill % equivalent to \vspace{\fill}
\clearpage
}

\newcommand{\answerspace}[1]{
  \begin{center}
    \textbf{BEGIN YOUR ANSWER BELOW THIS LINE} \\ \hrulefill \vspace{#1} \\ \hrulefill
  \end{center}
}

\newcommand{\answerspacefullpage}{
  \begin{center}
    \textbf{BEGIN YOUR ANSWER BELOW THIS LINE} \\ \hrulefill \pagebreak
  \end{center}
}

\newcommand{\additionalanswerspace}[1]{
  \begin{center}
    \textbf{CONTINUE YOUR ANSWER BELOW THIS LINE } \\ \hrulefill \vspace{#1} \\ \hrulefill
  \end{center}
}

\newcommand{\additionalanswerspacefullpage}{
  \begin{center}
    \textbf{CONTINUE YOUR ANSWER BELOW THIS LINE} \\ \hrulefill \pagebreak
  \end{center}
}

\newcommand{\freespace}[1]{
  \begin{center}
    \large \textbf{SCRAP SPACE BELOW} \\
    \hrulefill
    \pagebreak
  \end{center}
}

% Centered line
\newcommand{\mycenterline}[1]{
  \begin{center}
    \myline{#1}
  \end{center}
}

% Space for T/F:
\newcommand{\tfline}{\myline{.5cm}}

% For quick parenthesized and italicized point annotation.
\newcommand{\pts}[1]{{\em (#1 pts)}}
\newcommand{\onept}{{\em (1 pt)}}

% \item environments coupled with a line at the end, for students to write T and F in.
\newcommand{\tfitem}[1]{\item #1 \null\hfill \framebox(25,25){} \\ \hdashrule{0.95\textwidth}{1pt}{2pt}}
\newcommand{\setitem}[1]{\tfitem{$\curlybraces{#1}$} }
\newcommand{\lineitem}[2]{\item #1 \null \hfill \myline{#2}}

% Some circles and squares for students to fill in.
\newcommand{\whitecircle}[1]{\tikz[baseline=-0.5ex]\draw[black, radius=#1] (0,0) circle ;}
\newcommand{\whitesquare}[1]{\tikz\draw[black] (0,0) rectangl#1, #1) ;}

% Emphasis
\newcommand{\F}{$\mathbf{F}$}
\newcommand{\T}{$\mathbf{T}$}
\newcommand{\False}{\textbf{False}}
\newcommand{\false}{\textbf{false}}
\newcommand{\True}{\textbf{True}}
\newcommand{\true}{\textbf{true}}
\newcommand{\makered}[1]{\textcolor{red}{#1}}
\newcommand{\Rbbst}{\textcolor{red}{Red}-black tree}
\newcommand{\rbbst}{\textcolor{red}{red}-black tree}

\newcommand{\homeworkdata}[4]{
  \begin{mdframed}[linewidth=1pt]
    \noindent\makebox[\textwidth]{\LARGE \bf #1, #2 }
    \\\\
    \noindent\makebox[\textwidth]{\Large \bf  Homework \##3 }
    \\\\
    \noindent\makebox[\textwidth]{\large \bf  Due: #4}
    \\\\
    \noindent\makebox[\textwidth]{\large \bf Homework will not be accepted late}
  \end{mdframed}
  \vspace{40pt}
}

\usepackage{circuitikz}

\setlength{\parindent}{0em}
\setlength{\itemindent}{.5in}

\newcommand{\poneanswer}{%
}
\newcommand{\ptwoanswer}{%
}
\newcommand{\pthreeanswer}{%
}
\newcommand{\pfouranswer}{%
}
\newcommand{\pfiveanswer}{%
}
\newcommand{\psixanswer}{%
}
% \include{solutions}

%%%%%%%%%%%%%%%%%%%%%%%%%%%%%%%%%%%%%%%%%%%%%
%
%  STUDENTS - Your homework begins here.
%
%%%%%%%%%%%%%%%%%%%%%%%%%%%%%%%%%%%%%%%%%%%%%

\begin{document}
\pagestyle{empty}

\homeworkdata{CMSC 250}{Fall 2022}{6}{Sunday 23 Oct.\ 11:59pm}

{\Large \bf
  \begin{center}
  IMPORTANT
  \end{center}

  You can write your answers on any paper, either this paper
  or blank paper, or write your answer in Latex (template of this homework can be downloaded through ELMS).
  
  When you upload your document to Gradescope, make sure you tag your questions.

  \begin{center}
    YOU WILL NEED TO TAG YOUR PROBLEMS!!!
  \end{center}

  Problems which are not correctly found will not be graded, this is a zero-tolerance policy. 

  \begin{center}
    IF YOU ARE WORRIED...
  \end{center}

  If you have concerns about tagging your problems,
  We strongly suggest you drop by office hours and do it with a TA present so they can help you through the process,
  just to see how it works. In addition, Gradescope has a tutorial: \url{https://help.gradescope.com/article/ccbpppziu9-student-submit-work#submitting_a_pdf}


}

\pagebreak

\problem{Set Proof}{70}

\question{Suppose that $A, B, C$ and $D$ are sets, where $A \subseteq C$ and $B \subseteq D$. Is $A \times B \subseteq C \times D $ true? Prove or disprove the statement.}{14}
\bigskip
\bigskip
\textbf{Answer:} \\
Take an $(x, y)$ in $A \times B$. By the definition of $A \times B$, $x \in A \land y \in B$. Since $A \subseteq C$, $x \in C$. Since $B \subseteq D$, $y \in D$. Since $x \in C \land y \in D$, by the definition of cartesian product, $(x, y) \in C \times D$.\\
\,\\
Since $\forall (x, y) \in A \times B, (x, y) \in C \times D$, we get $A \times B \subseteq C \times D$.

\pagebreak

\question{Suppose that $A$ and $B$ are sets. Is $A \cap B = A \Rightarrow A \subseteq B$ true? Prove or disprove the statement.}{14}
\bigskip
\bigskip
\textbf{Answer:}
\counterwithout{equation}{section}
\begin{flalign}
    & A \cap B = A & \text{Assumption} & \\
    & A \cap B \subseteq A \land A \subseteq A \cap B & \text{Double containment (1)} & \\
    & A \subseteq A \cap B & \text{Specialization (2)} \\
    & \forall x \in A, x \in A \cap B & \text{Definition of $\subseteq$ (3)} \\
    & \forall x \in A, x \in A \land x \in B & \text{Definition of $\cap$ (4)} \\
    & \forall x \in A, x \in B & \text{Specialization (5)} \\
    & A \subseteq B & \text{Definition of $\subseteq$ (6)}
\end{flalign}
\setcounter{equation}{0}
Therefore, $A \cap B = A \Rightarrow A \subseteq B$ is true.

\pagebreak

\question{Prove that $\{x \in Z| (21 | x)\} \subseteq \{x \in  Z| (7 | x)\} $ }{14}
\bigskip
\bigskip
\textbf{Answer:}

\newcommand{\Z}{\mathbb{Z}}
To prove $\{x \in \Z \mid (21 \mid x)\} \subseteq \{x \in  \Z \mid (7 \mid x)\}$, we can prove $\forall x \in \{x \in \Z \mid (21 \mid x)\}, x \in \{x \in  \Z \mid (7 \mid x)\}$, or $\forall x \in \Z, 21 \mid x \Rightarrow x \in \{x \in  \Z \mid (7 \mid x)\}$, or $\forall x \in \Z, 21 \mid x \Rightarrow 7 \mid x$.
\begin{flalign}
    & 21 \mid x & \text{Assumption} & \\
    & \exists k \in \Z, x = 21k & \text{Definition of $\mid$ (1)} & \\
    & x = 7(3)k & \text{Factoring (2)} \\
    & 3k \in \Z & \text{$\Z$ closed over multiplication} \\
    & 7 \mid x & \text{Definition of $\mid$ (3, 4)}
\end{flalign}
\setcounter{equation}{0}
Since $\forall x \in \Z, 21 \mid x \Rightarrow 7 \mid x$, the original statement is also true.

\pagebreak


\question{Suppose that $A, B$ and $C$ are sets. Is $A \cap (B \cup C) \subseteq (A \cap B) \cup C$ true? Prove or disprove the statement.}{14}
\bigskip
\bigskip
\textbf{Answer:}

Need to prove or disprove $A \cap (B \cup C) \subseteq (A \cap B) \cup C$, which is logically equivalent to $\forall x \in A \cap (B \cup C), x \in (A \cap B) \cup C$.
\begin{flalign}
    & x \in A \cap (B \cup C) & \text{Assumption} & \\
    & x \in A \land x \in B \cup C & \text{Definition of $\cap$ (1)} \\
    & x \in A & \text{Specialization (2)} \\
    & x \in B \cup C & \text{Specialization (2)} \\
    & x \in B \lor x \in C & \text{Definition of $\cup$ (4)}
\end{flalign}

Case 1: $x \in B$
\begin{flalign}
    & x \in B & \text{Assumption} & \\
    & x \in A \land x \in B & \text{Conjunction (3, 6)} \\
    & x \in A \cap B & \text{Definition of $\cap$} \\
    & x \in A \cap B \lor x \in C & \text{Generalization (8)} \\
    & x \in (A \cap B) \cup C & \text{Definition of $\cup$}
\end{flalign}

Case 2: $x \not\in B$
\begin{flalign}
    & x \not\in B & \text{Assumption} & \\
    & x \in C & \text{Elimination (11, 5)} & \\
    & x \in (A \cap B) \lor x \in C & \text{Generalization (12)} \\
    & x \in (A \cap B) \cup C & \text{Definition of $\cup$}
\end{flalign}

In both cases, $x \in (A \cap B) \cup C$. This proves that $\forall x \in A \cap (B \cup C), x \in (A \cap B) \cup C$, which means the original statement is true.

\pagebreak

\question{Suppose that $A$ and $B$ are sets. Is $A - B = \emptyset \Rightarrow A \cap B^c = \emptyset$ true? Prove or disprove the statement without using Set Identities.}{14}
\bigskip
\bigskip
\textbf{Answer:}

Take sets $A$ and $B$.
\setcounter{equation}{0}
\begin{flalign}
    & A - B = \varnothing & \text{Assumption} & \\
    & A - B \subseteq \varnothing \land \varnothing \subseteq A - B & \text{Double containment} \\
    & A - B \subseteq \varnothing & \text{Specialization (2)} \\
    & \varnothing \subseteq A - B & \text{Specialization (2)}
\end{flalign}

\textbf{Proving $A \cap B^c \subseteq \varnothing$}:

Take an $x \in A \cap B^c$.
\begin{flalign}
    & x \in A \cap B^c & \text{Assumption} & \\
    & x \in A \land x \in B^c & \text{Definition of $\cap$ (5)} \\
    & x \in A & \text{Specialization (6)} \\
    & x \in B^c & \text{Specialization (6)} \\
    & x \not\in B & \text{Definition of complement (8)} \\
    & x \in A \land x \not\in B & \text{Conjunction (7, 9)} \\
    & x \in A - B & \text{Definition of set difference (10)} \\
    & x \in \varnothing & \text{Definition of $\subseteq$ (3, 11)}
\end{flalign}
Therefore, $\forall x \in A \cap B^c, x \in \varnothing$. This means that $A \cap B^c \subseteq \varnothing$.\\
\,\\
\textbf{Proving $\varnothing \subseteq A \cap B^c$}

Take an $x \in \varnothing$
\begin{flalign}
    & x \in \varnothing & \text{Assumption} & \\
    & x \in A - B & \text{Definition of $\subseteq$ (4, 13)} \\
    & x \in A \land x \not\in B & \text{Definition of set difference (14)} \\
    & x \in A & \text{Specialization (15)} \\
    & x \not\in B & \text{Specialization (15)} \\
    & x \in B^c & \text{Definition of complement (17)} \\
    & x \in A \land x \in B^c & \text{Conjunction (16, 18)} \\
    & x \in A \cap B^c & \text{Definition of $\cap$ (19)}
\end{flalign}
Therefore, $\forall x \in \varnothing, x \in A \cap B^c$. This means that $\varnothing \subseteq A \cap B^c$.\\
\,\\
Since $\varnothing \subseteq A \cap B^c$ and $A \cap B^c \subseteq \varnothing$, $A \cap B^c = \varnothing$.\\
\,\\
Therefore, $A - B = \varnothing \Rightarrow A \cap B^c = \varnothing$

\pagebreak



\problem{Set builder}{30}

Write the following sets in any of the accepted set notation we defined for this class.

\bigskip
\bigskip

\question{The set of all reals that are powers of three.}{6}

$\{x \in \mathbb{R} \mid \exists k \in \mathbb{Z}, 3^k = x\}$

\bigskip
\bigskip

\question{The set of all non-negative integers that are divisible by $4$ but are not a factor of 40!.}{6}

$\{x \in \mathbb{Z}^{\geq 0} \mid (4 \mid x) \land (x \nmid 40!)\}$

\bigskip
\bigskip

\question{The set of all points in 2D space (defined by a ordered pair) that are on the unit circle (a circle with radius 1) that are centered at the origin in the 2D Cartesian coordinate system.}{6}

$\{(x, y) \in \mathbb{R}^2 \mid x^2 + y^2 = 1\}$

\bigskip
\bigskip

\question{The set of all negative rationals such that their numerators are multiples of $2$ and denominators are multiples of $3$.}{6}

$\{\frac p q \in \mathbb{Q}^{< 0} \mid (2 \mid p) \land (3 \mid q)\}$

\bigskip
\bigskip

\question{The set that contains all rationals whose square root are irrational, and all integers whose powers are equivalent to $1$ in the mod $4$ class.}{6}

$\{x \in \mathbb{Q} \mid (\sqrt{x} \in \mathbb{R} - \mathbb{Q}) \lor (x \in \mathbb{Z} \land x \equiv 1 \pmod{4})\}$

\end{document}

%%% Local Variables:
%%% mode: latex
%%% TeX-master: t
%%% End:

