\documentclass[12pt, leqno]{article}
\usepackage[utf8]{inputenc}
\usepackage[margin=1in]{geometry}
\usepackage{amssymb}
\usepackage{amsmath}
\usepackage{parskip}

% Line before the therefore in proofs
\newcommand{\proofline}{\rule{0.75in}{0.5pt}}
% For set literals, wraps in {}
\newcommand{\set}[1]{\{#1\}}
% Powerset symbol
\newcommand{\powerset}{\mathcal{P}}
% Cardinality
\newcommand{\card}[1]{\lvert #1 \rvert}
\newcommand{\Z}{\mathbb Z}
\newcommand{\N}{\mathbb N}
\newcommand{\Q}{\mathbb Q}
\newcommand{\evens}{\Z^{\mathrm{even}}}
\newcommand{\odds}{\Z^{\mathrm{odd}}}
\newcommand{\Mod}[1]{\ (\mathrm{mod}\ #1)}

% permutations
\newcommand{\perm}[2]{{}_{#1}\mathrm{P}_{#2}}
% combinations
\newcommand{\comb}[2]{{}_{#1}\mathrm{C}_{#2}}

\title{Combinatorics}
\author{Yash Thakur}
\date{November 10, 2022}

\begin{document}
\counterwithout{equation}{section}

\maketitle

\subsection*{Some terms}

\textbf{Sample space:} the outcomes of something (for a coin flip, the sample space would be heads and tails)\\
\textbf{Event:} A subset of a sample space\\
\textbf{Random:} When it occurs, one outcome from a set of outcomes will occur, but it's impossible to know beforehand (a coin flip is a "random process")

The sample space of flipping a coin and rolling a die combined is \{(heads, 1), (tails, 1), (heads, 2), ..., (heads, 6), (tails, 6)\}. In general, if $A$ is the sample space of one event and $B$ is the sample space of another event, the sample space of both events is $A \times B$.

\textbf{Multiplication rule:} If $x$ and $y$ are independent events, with $A$ and $B$ being sample spaces respectively, the sample space of $x$ and $y$ is $A \times B$.

\subsection*{Permutations}

Number of permutations of size $r$ from set of size $n$: $\perm{n}{r} = \frac{n!}{(n - r)!}$

In a permutation, order matters.

\subsection*{Combinations}

In a combination, order doesn't matter. It's just a group/set. Combinations are like permutations without repeats.

$\comb{n}{r}$ means "using a group of size $n$, how many groups of size $r$ can be made?"

$\displaystyle \comb{n}{r} = \frac{\perm{n}{r}}{r!} = \frac{n!}{(n-r)!r!}$ (divide by $r!$ to get rid of duplicate groups)

\subsection*{Examples}

\subsubsection*{How many anagrams of "aaabb"?}

Want permutations but all a's are the same and b's are the same, so remove those. Divide by $3!$ for repeated a's, divide by $2!$ for repeated b's. Total anagrams: $\frac{\perm{5}{5}}{3!2!} = \frac{5!}{3!2!}$

\subsubsection*{How many ways can you choose and arrange 6 people around a table (there are 10 people in total)?}

The order in which you pick people matters, but since it's a table, rotating doesn't give you a new arrangement. So result is $\frac{\perm{10}{6}}{6}$

\subsubsection*{Have 6 friends, can only invite 4, and 2 refuse to be separated so have to invite either both or neither. How many ways to invite 4 friends?}

Combinations problem. Case 1: don't invite the pair, only 1 way to invite friends. Case 2: invite the pair, $\comb{4}{2} = 6$ ways to invite the rest. Total: 7.

\subsubsection*{How many different ways to buy 7 sodas if there are only 3 types of sodas?}

Think of it as putting two sticks between to divide into three groups. $\frac{9!}{2!7!} = 36$

\end{document}
